\chapter{Technical Background}
\label{sec:state}

% Hier werden zwei wesentliche Aufgaben erledigt:

% 1. Der Leser muß alles beigebracht bekommen, was er zum Verständnis
% der späteren Kapitel braucht. Insbesondere sind in unserem Fach die
% Systemvoraussetzungen zu klären, die man später benutzt. Zulässig ist
% auch, daß man hier auf Tutorials oder Ähnliches verweist, die hier auf
% dem Netz zugänglich sind.

% 2. Es muß klar werden, was anderswo zu diesem Problem gearbeitet
% wird. Insbesondere sollen natürlich die Lücken der anderen klar
% werden. Warum ist die eigene Arbeit, der eigene Ansatz wichtig, um
% hier den Stand der Technik weiterzubringen? Dieses Kapitel wird von
% vielen Lesern übergangen (nicht aber vom Gutachter ;-), auch später
% bei Veröffentlichungen ist "Related Work" eine wichtige Sache.

% Viele Leser stellen dann später fest, daß sie einige der Grundlagen
% doch brauchen und blättern zurück. Deshalb ist es gut,
% Rückwärtsverweise in späteren Kapiteln zu haben, und zwar so, daß man
% die Abschnitte, auf die verwiesen wird, auch für sich lesen
% kann. Diese Kapitel kann relativ lang werden, je größer der Kontext
% der Arbeit, desto länger. Es lohnt sich auch! Den Text kann man unter
% Umständen wiederverwenden, indem man ihn als "Tutorial" zu einem
% Gebiet auch dem Netz zugänglich macht.

% Dadurch gewinnt man manchmal wertvolle Hinweise von Kollegen. Dieses
% Kapitel wird in der Regel zuerst geschrieben und ist das Einfachste
% (oder das Schwerste weil erste).

%%%%%%%%%%%%%%%%%%%%%%%%%%%%%%%%%%%%%%%%%%%%%%%%%%%%%%%%%%%%%%%%%%%%%%%%%%%%%%%%
%                                                                              %
% TECHNICAL BACKGROUND                                                         %
%                                                                              %
%%%%%%%%%%%%%%%%%%%%%%%%%%%%%%%%%%%%%%%%%%%%%%%%%%%%%%%%%%%%%%%%%%%%%%%%%%%%%%%%



%%%%%%%%%%%%%%%%%%%%%%%%%%%%%%%%%%%%%%%%%%%%%%%%%%%%%%%%%%%%%%%%%%%%%%%%%%%%%%%%
%                                                                              %
% HIGH PERFORMANCE COMPUTING                                                   %
%                                                                              %
%%%%%%%%%%%%%%%%%%%%%%%%%%%%%%%%%%%%%%%%%%%%%%%%%%%%%%%%%%%%%%%%%%%%%%%%%%%%%%%%
\section{Cluster computing}
Parallel computers are not anymore specialized traditional
supercomputing platforms, instead low cost off-the-shelf commodity
computers form so called compute clusters \cite{ref:hpcc1}.

These loosely coupled components of a cluster are built from
multiprocessor PCs or workstation and are connected via a high-speed
network connection.

Therefore a compute cluster forms a homogeneous network of compute
entities, which are also called nodes. Every node can consist out of
different compute devices such as multicore CPUs and with more and
more popularity also acceleration devices like graphic cards. With the
consequence of a hierarchicle structure on the node with non-uniform
memory access (NUMA) \cite[numa]{ref:numa}.

The memory between nodes is not shared in principle, thus normally
every node has its own memory and therfore a compute cluster belongs
to the class of distributed memory architectures.

Only a few problems can be calculated without exchanging information
between other nodes. Thus each node has the challange to retrieve all
the data, that is needed to calculate the next step of its
computation.

\begin{itemize}
\item Computational power of single computer are limited by 
  phsical laws
\item Computers of a cluster need to work together
\item Future of high performance computing \cite{ref:hpc}
\item Scales
\end{itemize}

There are several ways to retrieve data from other nodes. The most
obvious way is to use direct communication between nodes. This
can be very low level protocols like TCP, UDP, IP by using UNIX sockets (\ref{sec:tcp_udp_ip}) or more modern and abstract models like the message passing interface (\ref{sec:mpi}) and the parallel virual machine (\ref{sec:pvm}).These traditional ways to communicate in 
cluster systems  will be discussed in section \ref{sec:communication}.

\todo{More related work + references to distributed shared memory} An
other way is to create a layer on top of the all cluster nodes, so
that the cluster looks like a single machine with a lot of devices and
a shared memory. This is called a distributed shared memory
architecture. The layer can be adjusted to the needs of the user.  It
can be an own operating system like MOSIX \cite{ref:mosix}, some
virtualisation layer \cite{ref:cluster_virt} etc.
This kind of data exchange between nodes will not be discussed in
detail in this thesis.
\begin{itemize}
\item Usage of simple parallization schemas like OpenMP, Pthreads
\item No explicit communication necessary
\item Easy usage for non computer scientists
\item higher abstraction \rightarrow loose of control and efficiency
\end{itemize}


%%%%%%%%%%%%%%%%%%%%%%%%%%%%%%%%%%%%%%%%%%%%%%%%%%%%%%%%%%%%%%%%%%%%%%%%%%%%%%%%
%                                                                              %
% TRADITIONAL COMMUNICATION MECHANISMS IN CLUSTER SYSTEMS                      %
%                                                                              %
%%%%%%%%%%%%%%%%%%%%%%%%%%%%%%%%%%%%%%%%%%%%%%%%%%%%%%%%%%%%%%%%%%%%%%%%%%%%%%%%
\section{Traditional communication mechanisms in cluster systems}
\label{sec:communication}


\subsection{IP,TCP, UDP, Sockets}
\label{sec:tcp_udp_ip}
\begin{itemize}
  \item Communication in the UNIX way
  \item IP, Internet Protocol was there before LAN and clusters \cite{ref:ip}
  \item TCP and UDP on top of IP (TCP/IP, UDP/IP) \cite{ref:tcp, ref:udp}
  \item TCP/IP and UDP/IP where made available through Berkeley Sockets \cite{ref:sockets}
  \item Sockets where designed by the UNIX philosophie; everything is a file (file descriptors)
  \item Sockets where not designed with the usage in hpc environment in mind
\end{itemize}

\subsection{Message passing intferface (MPI)}
\label{sec:mpi}

The message passing interface (MPI) is a standardized and portable
message-passing specification, specified by the Message Passing
Interface Forum \cite{ref:mpi_specification}.

The first version of the specification, known as MPI-1, was completed
in 1994 making use of the most attractive features of allready
existing message passing systems.  Further improvements followed 1997
by version MPI-2 and 2008 by version MPI-3. There might be some
confusion about the several MPI specification versions. Major versions
like MPI-2 or MPI-3 come with new features to the MPI standard and
minor versions like MPI-1.1 or MPI-1.2 come with clarifiacations of
allready existing features.  In the following the key features of the
varying MPI versions are listed:

\begin{itemize}
  \item MPI-1 (1994)
    \begin{itemize}
      \item Basix point-to-point communication
      \item Collectives
      \item Derived datatypes
    \end{itemize}
  \item MPI-2 (1997)
    \begin{itemize}
      \item Parallel I/O
      \item Remote memory operations (one-sided)
      \item Dynamic process management
    \end{itemize}
  \item MPI-3 (1998)
    \begin{itemize}
      \item Nonblocking collectives
      \item Improved one-sided communication interface
    \end{itemize}
\end{itemize}

An MPI programm is a set of autonomous processes, executing their own
code, in an MIMD style \cite{Flynn:1972:COE:1952456.1952459}.  Each
process rests in his own adress space and communicates via MPI
communication primitives with other processes in the same MPI
programm. 

Processes are identified according to their relative rank in a group by
consecutive integers from 0 to groupsizie-1. Which the initial group
compromises the full set of processes with ranks in the range 0 to 
the number of processes. A typical MPI program run is composed from
the following steps:

\begin{itemize}
  \item MPI process
    \begin{enumerate}
      \item Start MPI programm
      \item Init MPI environment
      \item Ask for own rank
      \item Ask for number of ranks overall
      \item Depend on own rank and overall ranks do some task (varying flow of control between ranks)
      \item Communicate with other ranks
      \item Finalize MPI environment
      \item Exit MPI programm
    \end{enumerate}
\end{itemize}

MPI provides a large and versatile set of routines. At its most
basic, MPI provides function for sending and receiving a messages.
Both blocking
\begin{itemize}
  \item $MPI\_Send$
  \item $MPI\_Recv$
\end{itemize}
and nonblocking
\begin{itemize}
\item $MPI\_ISend$
\item $MPI\_IRecv$
\end{itemize}

\begin{itemize}
\item Standardized and portable message-passing \cite{ref:mp} specification, not implementation
\item High level of abstraction
\item layer 5 and higher of OSI Reference Model
\item Fits needs of parallel programming community
\item Sending messages between processes on same or different nodes
\item high performance at a relatively low cost

\item Communication on top of existing communication mechanisms
  (overlay network, communication middleware)
\item De facto standard for communication in cluster / hpc systems
\item Implementations are available on virtually every parallel
  computer systems
\item Overlay network implemented on top of the communication
  abstractraction provided by the system (sockets, AM, etc.)
\item Zero-data copy transfer
\item Usable in shared and distributed memory systems
\item Free libraries available (OpenMPI)
\item Language independent communication protocol
\end{itemize}

\subsection{Parallel virtual machine (PVM)}
\label{sec:pvm}
\begin{itemize}
\item Parallel Virtual Machine connects a collection of heterogeneous computers
  to a single "parallel virtual machine"
  (http://en.wikipedia.org/wiki/PVM)
\item Support for communication and synchronization operations
\item Configuration control
\item synamic Spawning of pocressen
\item PVM daemons are spawned on a set of nodes
\item Local processes connect to PVM daemons and can 
  communicate through this daemon to other PVM daemons
\end{itemize}

\subsection{Overlay / P2P networks}
\section{Fault tolerance}
\section{Load balancing}

%%%%%%%%%%%%%%%%%%%%%%%%%%%%%%%%%%%%%%%%%%%%%%%%%%%%%%%%%%%%%%%%%%%%%%%%%%%%%%%%
%                                                                              %
% PIConGPU - THE APPLICATION                                                   %
%                                                                              %
%%%%%%%%%%%%%%%%%%%%%%%%%%%%%%%%%%%%%%%%%%%%%%%%%%%%%%%%%%%%%%%%%%%%%%%%%%%%%%%%
\section{PIConGPU - the application}
\label{sec:picongpu}
\begin{itemize}
  \item A Many-GPGPU Particle-in-Cell Code
  \item Electric and magnetic fields are interpolated on a physical
    grid dividing the simulated volume into cells
  \item Simulated volume divided in cubes / squares
  \item Every GPU computes one of these cubes / squares
  \item One MPI process per cube square
  \item Offload of simulation data to GPUs
  \item Communication based on MPI
\end{itemize}

\subsection{Requirements for the upcoming generation of super computers}
\begin{itemize}
\item Run picongpu on xeon phi in native mode
\item Exchangeable communication layer to be portable / ready  for the upcoming
  generation of acceleration devices
\end{itemize}

\subsection{Communication topologie}
\begin{itemize}
\item 3- or 2-dimensional mesh
\item next neighbor communication
\item Some collective operations (Allgather, Gather, GatherV, AllReduce, Reduce
\item Offload of simulation data to GPU
\end{itemize}


\cleardoublepage

%%% Local Variables:
%%% TeX-master: "diplom"
%%% End:
