\chapter{Design}
\label{sec:design}

% Ist das zentrale Kapitel der Arbeit. Hier werden das Ziel sowie die
% eigenen Ideen, Wertungen, Entwurfsentscheidungen vorgebracht. Es kann
% sich lohnen, verschiedene Möglichkeiten durchzuspielen und dann
% explizit zu begründen, warum man sich für eine bestimmte entschieden
% hat. Dieses Kapitel sollte - zumindest in Stichworten - schon bei den
% ersten Festlegungen eines Entwurfs skizziert werden.
% Es wird sich aber in einer normal verlaufenden
% Arbeit dauernd etwas daran ändern. Das Kapitel darf nicht zu
% detailliert werden, sonst langweilt sich der Leser. Es ist sehr
% wichtig, das richtige Abstraktionsniveau zu finden. Beim Verfassen
% sollte man auf die Wiederverwendbarkeit des Textes achten.

% Plant man eine Veröffentlichung aus der Arbeit zu machen, können von
% diesem Kapitel Teile genommen werden. Das Kapitel wird in der Regel
% wohl mindestens 8 Seiten haben, mehr als 20 können ein Hinweis darauf
% sein, daß das Abstraktionsniveau verfehlt wurde.

%%%%%%%%%%%%%%%%%%%%%%%%%%%%%%%%%%%%%%%%%%%%%%%%%%%%%%%%%%%%%%%%%%%%%%%%%%%%%%%%
%                                                                              %
% MOTIVATION AND REQUIREMENTS                                                  %
%                                                                              %
%%%%%%%%%%%%%%%%%%%%%%%%%%%%%%%%%%%%%%%%%%%%%%%%%%%%%%%%%%%%%%%%%%%%%%%%%%%%%%%%
\section{Motivation}
\begin{itemize}
\item Parallel computers are stricly necessary in scientific community
\item Parallel computers are compute clusters \ref{sec:cluster}
\item In a nutshel very homogeneous compute architecture
\item Compute nodes connected by abitrary network
\item Domain decomposition \ref{sec:domain_decomposition} of physic
  simulation and static mapping to compute architecture
\item Simulation is changing over time and load may be not distributed
  anymore
\item PIConGPU \ref{sec:picongpu} is even harder because of an
  hierarchicle memory architecture.
\item NUMA architecture will crow even harder
\end{itemize}

\subsection{Requirements}
To be able to model such complex and heterogeneous topologies the
following components are required:

\begin{itemize}
\item Communication abstraction layer, which introduce an abstraction
  from existig communication layers.
\item Description of the communication topology
\item Possibility to communicate inside the physical domain
\item Mapping of communiation topology onto the hardware topology
\end{itemize}

%%%%%%%%%%%%%%%%%%%%%%%%%%%%%%%%%%%%%%%%%%%%%%%%%%%%%%%%%%%%%%%%%%%%%%%%%%%%%%%%
%                                                                              %
% ANALYSIS OF PIConGPU CODE                                                    %
%                                                                              %
%%%%%%%%%%%%%%%%%%%%%%%%%%%%%%%%%%%%%%%%%%%%%%%%%%%%%%%%%%%%%%%%%%%%%%%%%%%%%%%%
\section{Analysis of PIConGPU code}
\begin{itemize}
  \item How does the communication topologie looks like ?
  \item What are the different levels of communication ?
  \item What kind of communcation layer is used ?
  \item How is Communication implemented in PIConGPU ?
\end{itemize}

%%%%%%%%%%%%%%%%%%%%%%%%%%%%%%%%%%%%%%%%%%%%%%%%%%%%%%%%%%%%%%%%%%%%%%%%%%%%%%%%
%                                                                              %
% COMMUNICATOR                                                                 %
%                                                                              %
%%%%%%%%%%%%%%%%%%%%%%%%%%%%%%%%%%%%%%%%%%%%%%%%%%%%%%%%%%%%%%%%%%%%%%%%%%%%%%%%
\section{Abstraction from existing communication layers - Communicator}
\todo{mention policy based design}
\todo{Wie trägt komponente bei motivation zu erfüllen}
The abstraction from existing communication layers is a fundamental
property for a flexibel and portable application. Compute clusters are
not the same in general. Indeed, there are existing standards for
communication in cluster system, but the world of computer science is
subject to constant changes. Thus the possibility to exchange the
underlying communication layer without changing the interface to the
communication inside the application, is a prerequisite for future
applications in distributed computing.

The challange is to provide a very general interface, that is able to
map varying communication libraries onto it. This interface should
remind the programmer on known communication schemas, which does not
need a rethink.

The communicator is the connection to a network of an existing
communication layer. The user can create several communicators with
varying communication layers underneath. Or it is also possible to
create several communicators pointing into the same network.

Every communicator is identifiable through identifier, that is unique
inside the network in which the communicator is connected to.

In first place the communicator provides peer-to-peer
communications. That are basic communication operations to send and
receive abitrary data between two communicators. These operations are
available blocking and also non-blocking.  The interface is influenced
by existing communication libraries, that share some common interface.

\begin{itemize}
  \item Data \\ Usually, this is a pointer to some memory.  But can
    also be implemented with references and c++ templates.  This makes
    it possible to send some abitrary datatypes
  \item DataSize\\ Has to be transmitted when only a pointer is
    given. In the case that some stl container like vector or arrays
    is handed over, the size can be determined by the size() function
  \item Destination / Source This is a adress of the receiving or
    sending process.
    
\end{itemize}

To support also MPI as backend, the following information have to be
added to the communication interface:

\begin{itemize}
  \item Tag Is used in MPI to seperate messages from the process and
    give them meaning without the need of an message header or
    something similar
  \item Context This gives MPI the information which processes are
    able to communicate with each other.
    
\end{itemize}

Hint: Context and tag could be removed, thus some default tag and a
global context would be choosen for the operations.



\begin{itemize}
\item There are many communication libraries
\item The challange is to provide a very general interface
\item Comparing several communication libraries like MPI, ZMQ,
  boost::asio and boost::mpi there is some common interface for
  peer-to-peer communication.
\item send(data, dataSize, destination)
\item recv(data, dataSize, source)
\item It provides also collective operations known from MPI
\end{itemize}

\begin{itemize}
\item Peer to peer operations
  \begin{itemize}
  \item send
  \item recv
  \item asyncSend
  \item asyncRecv
  \end{itemize}
\end{itemize}

\begin{itemize}
\item Collective operations
  \begin{itemize}
    \item gather
    \item gather2
    \item allGather
    \item allGather2
    \item reduce
    \item allReduce
    \item broadcast
    \item syncronize
    \item createContext
  \end{itemize}

\end{itemize}

%%%%%%%%%%%%%%%%%%%%%%%%%%%%%%%%%%%%%%%%%%%%%%%%%%%%%%%%%%%%%%%%%%%%%%%%%%%%%%%%
%                                                                              %
% GRAPH                                                                        %
%                                                                              %
%%%%%%%%%%%%%%%%%%%%%%%%%%%%%%%%%%%%%%%%%%%%%%%%%%%%%%%%%%%%%%%%%%%%%%%%%%%%%%%%
\section{Description of the communication topologie - Graph}
The considerations started with the analysis of the domain composition
of the PIConGPU simulation volume. The PIC code has several levels of
decompositions. First there is the partitioning into small cells,
which is the real partitioning of the physical volume. These cells are
the smallest entities of the simulation.

Further cells are combined to supercells, which is a memory optimized
set of cells.

And finally are supercells combined to a cluster of supercells which
than are calculated by a single compute device. So the first
considerations where to model exactly these three levels: physical
topologie, memory topologie, communication topologie.

But the mentioned topologie follow all the same principles. They
connect some entities. Therefore every topologie is possible to map
onto a graph.  So a graph can be used to describe several levels of
doman decomposition with a single language.

Properties of the graph
\begin{itemize}
  \item Directed graph
  \item Multiple edges (multi graph)
  \item Allows loops
\end{itemize}

%%%%%%%%%%%%%%%%%%%%%%%%%%%%%%%%%%%%%%%%%%%%%%%%%%%%%%%%%%%%%%%%%%%%%%%%%%%%%%%%
%                                                                              %
% NAMESERVICE                                                                  %
%                                                                              %
%%%%%%%%%%%%%%%%%%%%%%%%%%%%%%%%%%%%%%%%%%%%%%%%%%%%%%%%%%%%%%%%%%%%%%%%%%%%%%%%
\section{Mapping onto the hardware}

%% Requirements
Fact is, that the communication hardware layout is fixed. The access
to the hardware is represented by communicators and these are
connected in the simplest case by an fully connected network, thus
every communicator can send messages to every other
communicator. Therefore a network on top of the communicators has to
be established.

It is a very important property to be able to explicitly map the
communication topologie of the application onto the communication
hardware.

In the first place it should gives the application the possibility to
be optimized on special charakteristics of the compute cluster. For
example could vertices that need to communicate very intensively with
each other can be mapped on neighboring nodes or even onto the same
node. Which makes communication more efficient.

Besides it should be possible to change the mapping of the
communication topologie onto the communication hardware in
runtime. This is important when thinking of occuring unbalanced load
during the run of the application.

%% Process of announce
The objective is to establish a communication overlay on top of the
network provided by the communicators. This overlay network is based
on a graph. Thus all communicators that want to take part on the
communicatin in this overlay need to know this graph.

This Graph can be constructed in parallel by all processes, delivered
by a distributed file system or could even be delivered by some
construction process by explicit communication operations. That is
left open to the user of this library.

The next phase is the distribution of the vertices of the graph to the
processes. This has the meaning that a process responsible for the
communication of this processes. The process will be called host of
the vertices.

The distribution can be done by several methods: randomized, round
robin, dictated by some master process etc. This distribution is again
left open to the user.

Every communicator hosts zero, one or more vertices, so to speak it is
also possible that one communicator host all vertices and communicates
in the end with itself.  Thus the processes announce their hosted
vertices to all other communicators, which will register this
information into their Vertex->Communicators map.  This can be done by
the collective operation allGatherV to the network of communicators.


The NameService is the connection of the graph as the topology
description instance and the communicator as communication hardware
interface.

It contains two mappings. Vertices of a graphs are mapped to
Communicators and Graphs are mapped to contexts.

%% Remapping of vertices
Because the communication hardware and topolgie is seperated,
it is also possible to change the mapping from vertices to 
communicators on runtime. That can be necessary when load
of an application is getting inbalanced during the runtime.

The application can perform a global load balancing step, which
is very similar to static load balancing in the beginning
of the application. Thus vertices are seperated by some rule
and then announced. The difference is, that the communicators
have to send the data of their once hosted vertices to the
new hosts of the vertices.

Another possibility is to do a kind of local load balancing.
This means that communicators hand over vertices to communicators
of adjacent vertices. This might not require a global synchronization
step.







\begin{itemize}

\end{itemize}

%%%%%%%%%%%%%%%%%%%%%%%%%%%%%%%%%%%%%%%%%%%%%%%%%%%%%%%%%%%%%%%%%%%%%%%%%%%%%%%%
%                                                                              %
% GRAPHCOMMUNICATOR                                                            %
%                                                                              %
%%%%%%%%%%%%%%%%%%%%%%%%%%%%%%%%%%%%%%%%%%%%%%%%%%%%%%%%%%%%%%%%%%%%%%%%%%%%%%%%
\section{Communication on physical domain - GraphCommunicator}

\begin{itemize}
\item Communication on the graph level
\item Vertices are peers of the communication network and can be adressed directly
\item Vertices of a graph are able to exchange data with each other
\item All vertices of a graph can execute collective operations

\item Direct connection Relatively stupid, just connection of communicator and nameService
\begin{itemize}
\item 
\end{itemize}

\item More logic in collective operations
  \begin{itemize}
    \item Gather\\
      Vertices on the same host collect data local first and then use
      the collective gather provided by the communicator
    \item Reduce\\
      Vertices on the same host reduce data local first and then use
      reduce provided by the communicator.
    \item Synchronize\\
      This is a barrier function which lets the host wait for all others host
      that are contributing to a special graph. 
    \item There is some problem with these collective operations when a host
      forgets to call the collective on all its hosted vertices of an graph.
      Then the collective operation can not be finished and the application 
      is therefore locked. This problem is passed wo wise programmers which
      do not forget this issue.
  \end{itemize}
  

\item Implemented on top of communicator as overlay network
\item Location information of vertices retrieved from nameService
\end{itemize}



%%%%%%%%%%%%%%%%%%%%%%%%%%%%%%%%%%%%%%%%%%%%%%%%%%%%%%%%%%%%%%%%%%%%%%%%%%%%%%%%
%                                                                              %
% ETC                                                                          %
%                                                                              %
%%%%%%%%%%%%%%%%%%%%%%%%%%%%%%%%%%%%%%%%%%%%%%%%%%%%%%%%%%%%%%%%%%%%%%%%%%%%%%%%

\subsection{Requirements for the upcoming generation of super computers}
\begin{itemize}
\item Run picongpu on xeon phi in native mode
\item Exchangeable communication layer to be portable / ready for the
  upcoming generation of acceleration devices
\end{itemize}


The during this thesis developed communication layer was leaded by the
needs of PIConGPU as an example application for high performance
computing on cluster systems.

The implementation of PIConGPU has some lack of abstraction for their
communication. Because there is no abstraction layer which hides the
MPI communication calls, which makes it impossible to exchange it.

Also the communication topology of the simulations is not be modeled
satisfying. Neighboring subdomains are directly adressed by their
ranks, thus changing the communication topology also leads to a lot of
changes in the algorithm.



\cleardoublepage

%%% Local Variables:
%%% TeX-master: "diplom"
%%% End:
