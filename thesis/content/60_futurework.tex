\chapter{Future Work}
\label{sec:futurework}

\begin{itemize}

\item Implementation of further adapters
  \begin{itemize}
  \item Adapter for CUDA
  \item Adapter for ZMQ
  \item Adapter for Boost:asio
  \end{itemize}

\item Implementation of multiadapter design
  
  Is maybe not the best idea. GVON should be maybe by design
  bounded to only a single adapter. Discussion about such
  a system is okay, but should not to be future work by need.
  \begin{itemize}
  \item Implementing grid system on basis of multiadapterdesign
  \item Support of collective operations over the borders
    of different adapters in the case that the CAL supports more than one adapter
    being instanciated.
  \item implement routing between multiadapters
  \end{itemize}

\item Thread safe implementation of the gvon collectives

\item Deployment in real world applications like PIConGPU

\item Fault tolerance through multi vertex map

\item Graph description of adapter hardware topologie
  \begin{itemize}
  \item Graph would describe physical connections between peers
  \item Properties could describe bandwidth, latency etc. 
  \item Instead mapping drom vertex to independent peers, mapping
    from vertex to peers related to a graph
  \item Usage of graph partitioning algorithm (metis, parmetis etc.)
  \item static, dynamic, autotuning of mapping
  \item basis for load balancing
  \end{itemize}

\item Deployment in real world application
  \begin{itemize}
  \item PIConGPU
  \item HASEonGPU
  \end{itemize}

\item Compile time generation of static graphs

\item Optimization of the graph to reduce overhead

\end{itemize}

\cleardoublepage

%%% Local Variables:
%%% TeX-master: "diplom"
%%% End:
