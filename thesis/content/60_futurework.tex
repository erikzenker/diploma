\chapter{Future Work}
\label{sec:futurework}



%%%%%%%%%%%%%%%%%%%%%%%%%%%%%%%%%%%%%%%%%%%%%%%%%%%%%%%%%%%%%%%%%%%%%%%%%%%%%%%%
%                                                                              %
% MORE ADAPTERS                                                                %
%                                                                              %
%%%%%%%%%%%%%%%%%%%%%%%%%%%%%%%%%%%%%%%%%%%%%%%%%%%%%%%%%%%%%%%%%%%%%%%%%%%%%%%%
\section*{Further Adapter implementations}


\begin{itemize}
\item The prototype implementation (Section~\ref{sec:implementation})
  utilized MPI as communication backend. While, MPI is probably
  available on every computing system and implements probably every
  important communication protocol, also other interesting
  communication libraries could be used as communication backend.

  % Internet Socket based
\subsection*{Internet Socket Based Adapter}
\item Especially distributed system,that are not deployed on clusters,
  usually use other communication libraries than MPI.

\item Two interesting libraries are ZMQ and Boost Asio, which are
  both based on the TCP/IP and UDP/IP protocol. They can be used to
  interconnect applications over the internet. Therefore,
  computations using the internet as network are imagineable.

\item Implementing an adapter that interfaces with ZMQ or Asio would
  open an application based on the GVON interface to the world of
  distributed computing over the internet.
  
\item Examples for such applications are folding@home
  \cite{ref:folding_at_home} and seti@home
  \cite{ref:seti_at_home}. These application distribute a complex
  problem to a big number of peers over the internet.

\item A further use case is the connection of cluster nodes over the
  internet. This would construct a grid computing environment based on
  the GVON interface. It could easily extend the computational power
  of a cluster by further computing units. Whereby, all peers utilize
  the same communication interface.

  % Adapters for accelerators
\subsection*{Accelerator Based Adapter}
\item In contrast to communication over a big network such as the
  internet, is the modeling of the communication with and in between
  accelerator devices a more local issue. An adapter could model the
  mechanism of accelerator communication for CUDA or OpenCL.

\item Since, offload is a one-sided communication concept, the
  accelerator device needs to be managed from a host CPU. This host
  needs to transform the one-sided communication into a two-sided
  communication that is supported by the developed system.

\item Communication processes, such as send and receive, between
  accelerators are managed by their hosts. Therefore, the pair of host
  and device form a peer of the CAL. The transmission of data from one
  accelerator A to another accelerator B is separated into the
  following steps

  \begin{enumerate}
  \item Copy of data from A to host of A
  \item Transmission of data from host of A to host of B
  \item Copy of data from host of B to B
  \end{enumerate}

  % MPI on CUDA devices
\item This adapter for accelerators has the drawback that data has to
  take the detour above the host CPU. The emerge of CUDA version 4.0
  and 5.0 provides techniques that remove the need for the detour.

\item First of all, introduces CUDA 4.0 the Unified Virtual Addressing
  (UVA).  It creates a uniform address space of and devices of a
  single node.

\item The introduction of GPUDirect offers a direct exchange of data
  between accelerators on the same node (P2P) and of data between
  accelerator connected by a network (RDMA).  Both approaches bypass
  the CPU and exchange data directly over the PCI bus or network
  controller.

  \todo{Graphic of CUDA with GPUDirect}

  \todo{Provide small example}

\end{itemize}


%%%%%%%%%%%%%%%%%%%%%%%%%%%%%%%%%%%%%%%%%%%%%%%%%%%%%%%%%%%%%%%%%%%%%%%%%%%%%%%%
%                                                                              %
% MULTIPLE ADAPTER                                                             %
%                                                                              %
%%%%%%%%%%%%%%%%%%%%%%%%%%%%%%%%%%%%%%%%%%%%%%%%%%%%%%%%%%%%%%%%%%%%%%%%%%%%%%%%
\section*{Multiple Adapter Design}
\sitem{Until now, the adapter design is chosen deliberately}

But, the possibility to exchange the adapter of the CAL also raises
the question if a design with more than one adapter at the same time
would be possible. Connecting the network of at least two adapters
would form a heterogeneous network, where each network has its own
properties like latency, bandwidth and hardware topology. The CAL
would unify these varying networks transparently under the same
interface. Thus, a multi adapter design is the foundation for the
usage of the developed system in jungle computing environments
\ref{sec:jungle}.


\sitem{A CAL with multiple adapters could connect two cluster systems
  over internet.}

\sitem{This would create a powerful tool to extend the computing power
  of a single cluster}

\sitem{Easily create grid computing structures}

Assume, the CAL has two adapters, one for MPI and another for sockets,
to chose from and some peers of the network are able to communicate
through both adapters.  The CAL needs to provide a list of all
avaiblabe adapters for a specific peer. A particular sorting of this
list could state out the prefered adapter for communication between
two peers. Exchanging data between peers always requires a lookup of
which adapter needs to be selected to address the peers.

\sitem{On the implementation side, there are two solution to switch
  between adapters at run-time}

\sitem{strategy patter, is the run-time solution}

\todo{Inform about strategy pattern}

\sitem{variadic templates, is the compile-time solution}

Furthermore, varying real addresses spaces of different adapters have
to be mapped onto the same unified address space of the CAL.  It opens
up the possibility that two peers which do not share a same adapter,
but are connected over a third adapter, could still exchange data by
routing over that third adapter. Thus, a multi adapter design does
only make sense when routing between two adapters is implemented.

\sitem{Same problem with collective operations between varying
  adapters}

\sitem{Support of collective operations over the borders of different
  adapters in the case that the CAL supports more than one adapter
  being instantiated.}

Even if the design of a multi adapter communication abstraction layer
looks quite interesting, the implementation is rather more complex
than a single adapter design. The decision for a single or multiple
adapter CAL should be placed in the hand of the application developer.
Thus, the developer could configure the CAL by a further policy that
defines the adapter behavior.


%%%%%%%%%%%%%%%%%%%%%%%%%%%%%%%%%%%%%%%%%%%%%%%%%%%%%%%%%%%%%%%%%%%%%%%%%%%%%%%%
%                                                                              %
% DEPLOYMENT IN REAL WORLD SIMULATION                                          %
%                                                                              %
%%%%%%%%%%%%%%%%%%%%%%%%%%%%%%%%%%%%%%%%%%%%%%%%%%%%%%%%%%%%%%%%%%%%%%%%%%%%%%%%
\section*{Deployment in Real World Simulations}
\begin{itemize}

\item Section~\ref{sec:impl:gol} and \ref{sec:impl:nbody} have shown
  that implementations of Game of Life and N-body simulations on basis
  of the GVON are possible. Furthermore, the performance evaluation in
  Section~\ref{sec:eval:real} have shown no significant overhead in
  respect to equivalent MPI implementations.

  \subsection*{Deployment in PIConGPU}
\item PIConGPU was analyzed for the needs of a flexible communication
  approach, but was not objective for deployment.

\item The next step for evaluating the developed system is the
  deployment of the GVON as library into PIConGPU. This will show how
  well the developed system can be integrated into existing projects.


\item The PIConGPU topology is based on a three-dimensional grid with
  diagonal connections. It does not differ to much from communication
  patterns used in the GoL simulation.

\item Thus, a replacement of communication related part in PIConGPU
  source code should be easily possible.

\item PIConGPU is also interesting with respect of an CUDA-aware
  adapter.  Most calculation in PIConGPU are offloaded to the
  CPUs. Therefore, bypassing the CPU in case of device to device
  copies would increase overall performance.

  % HASEonGPU
  \subsection*{Deployment in HASEonGPU}
\item HASEonGPU is another project developed at the Helmholz Zentrum
  Dresden Rossendorf.  It was never an object of communication code
  analyses, but it would be interesting if the developed communication
  approach would easily fit on HASEonGPU.

\item HASEonGPU is Monte-Carlo simulation of photons in a laser gain
  medias. Whereby, the amplified spontaneous emission (ASE) of photons
  is calculated for certain sample points in the gain media.

\item The ASE value of each sample point can be calculated
  independently from each other. Thus, a distribution onto a cluster
  is possible.

\item In HASEonGPU a master peer distributes sample points onto
  available worker peers. This forms a communication based on a
  star-topology, where the master peer is placed as star center. In a
  scenario with a lot of workload, e.g. by lot of sample points, the
  master peer would be the communication bottleneck.
  

\item But, the utilization of the GVON would make an exchange of this
  topology very easy. The star-topology could be replaced by a more
  general tree topology. The master peer could delegate distribution
  task to sub-master peers that are responsible for sub-trees.
\end{itemize}

%%%%%%%%%%%%%%%%%%%%%%%%%%%%%%%%%%%%%%%%%%%%%%%%%%%%%%%%%%%%%%%%%%%%%%%%%%%%%%%%
%                                                                              %
% GRAPH PARTITIONING                                                           %
%                                                                              %
%%%%%%%%%%%%%%%%%%%%%%%%%%%%%%%%%%%%%%%%%%%%%%%%%%%%%%%%%%%%%%%%%%%%%%%%%%%%%%%%
\section*{Graph Partitioning}
\begin{itemize}

  % Load balancing
\item In the beginning of this work it was clarified, that load
  balancing is not the topic of this work. Therefore, Only rudimentary
  distribution algorithms such round robin and consecutive were
  implemented. But, it was also claimed, that it should be possible to
  build load balancing on top.

\item This and the following Section will discuss ideas for load
  balancing based on the developed system.

\item Assume, a simulation domain is modeled by an graph and the graph
  has more vertices than available peers in the global context. Peers
  need to be oversubscribed with vertices in this situation.

\item Therefore, hosted vertices on a host should be as close
  connected as possible with respect to the graph. So that,
  communication between vertices is as local as possible. This
  potentially minimizes the amount of messages that have to be
  transferred over the network.

% Graph partitioning
\item A well researched solution for this problem is partitioning of
  graphs. Partitioning divides the a graph into smaller components.
  A good partition is defined as one in which the number of edges
  running between separated components is small.

\item Since the communication topology is described by a graph, it can
  be easily partitioned by already existing graph partitioning tools.

  % Metis
  \subsection[Metis as Graph Partitioning Tool}
\item The tool METIS and its related tools hMETIS and ParMETIS are
  established applications for graph partitioning.T hey provide high
  quality partitions and are two magnitudes faster than other widely
  used partitioning algorithms compared to other graph partitioning
  tools.

\item The developers claim that Graphs with several millions of
  vertices can be partitioned in 256 parts in a few seconds on current
  generation workstations and PCs

\item Published under the Apache Licence Version 2.0. Therefore, Metis
  could be added to the graph distribution functions

  \todo{Which format does Metis need}
  \todo{How can it be deployed in existing software}
  
\end{itemize}

%%%%%%%%%%%%%%%%%%%%%%%%%%%%%%%%%%%%%%%%%%%%%%%%%%%%%%%%%%%%%%%%%%%%%%%%%%%%%%%%
%                                                                              %
% DESCRIPTION OF ADAPTER HARDWARE TOPOLOGY                                     %
%                                                                              %
%%%%%%%%%%%%%%%%%%%%%%%%%%%%%%%%%%%%%%%%%%%%%%%%%%%%%%%%%%%%%%%%%%%%%%%%%%%%%%%%
\section*{Description of the Adapter Hardware Topology}
\begin{itemize}

% Cluster network topology
\item Cluster systems are equipped with a varying network
  systems. Usually with custom build network topologies.  Utilizing of
  knowledge about the network topology can increase application
  performance

\item The most general approach to describe the network topology of a
  cluster is a graph. A graph would describe physical connections
  between nodes and could be annotated with latency and bandwidth
  information.

\item This graph could be the foundation of varying graph
  algorithm. For example could the distance between two nodes be
  estimated and utilized to reduce communication latency between
  peers.
  
\item With the existence of two graphs, one modeling the communication
  topology of the simulation and another modeling the network
  topology, a mapping between these graphs is possible.
  
\item Instead a mapping from vertices to independent peers, vertices
  can be mapped onto peers modeled in the graph. Information that were
  used to describe the network can be used to optimize this mapping.

\item In an oversubscribed case, could the communication graph first
  be partitioned to the number of available peers. After that, the
  partitions will be mapped onto the peers.

\item An optimal mapping would result in an graph
  homomorphism. Deciding whether there exists a homomorphism from one
  graph to another, is NP-complete.Therefore, a heuristically approach
  that tries to maximize the amount of same adjacent vertices in
  communication and network graph should be chosen
  
% Load balancing
\item Mapping of the communication graph onto the network graph opens
  new possibilities
\item First of all a static mapping can be performed

  % Static load balancing
\item Static mapping takes already known information about the
  simulation and network into account, before execution
\item But the simulation can change their load distribution during
  run-time

  % Dynamic load balancing
\item Dynamic redo load balancing at run-time
\item load balancing at fixed points of code execution
\item every n time-steps

  % Auto tune
\item Create algorithm to auto tune load balancing
\item Generate a metric to evaluate balancing of simulation
\item System recognizes load imbalance and start
  load balancing step automatically
\end{itemize}


%%%%%%%%%%%%%%%%%%%%%%%%%%%%%%%%%%%%%%%%%%%%%%%%%%%%%%%%%%%%%%%%%%%%%%%%%%%%%%%%
%                                                                              %
% IDEAS FOR FAULT TOLERANCE                                                    %
%                                                                              %
%%%%%%%%%%%%%%%%%%%%%%%%%%%%%%%%%%%%%%%%%%%%%%%%%%%%%%%%%%%%%%%%%%%%%%%%%%%%%%%%
\section*{Ideas for Fault Tolerance}
\begin{itemize}
\item No Fault tolerance techniques were implemented so far

\item The developed system makes it possible to build
  fault tolerance on top

\item Failure of peers during execution of simulation not impossible
\item Especially for increasing amount of computing power
\item In best case no re-execution of full simulation necessary


% Reload from checkpoint
\item Failure peer has created check point on regular basis onto hard disk
\item Loading of check point is possible
\item Other peer can load checkpoint and take over vertices of failure
  peer
\item Peer announces its new vertices
\item The other peer could already be a host
\item Or it could be a special backup peer that usually
  is not part of the simulation calculations
\item after loading the state and re-announce of the vertices
  calculations can go on


% Redundant calculations
\item Another possibility without the need for reloading the check point
  is the redundant calculations of simulation subdomains
\item A vertex can be distributed to several peers
\item The system ensures that data receives all peers
\item Usage of multi-maps

\end{itemize}

%%%%%%%%%%%%%%%%%%%%%%%%%%%%%%%%%%%%%%%%%%%%%%%%%%%%%%%%%%%%%%%%%%%%%%%%%%%%%%%%
%                                                                              %
% TWEAKS                                                                       %
%                                                                              %
%%%%%%%%%%%%%%%%%%%%%%%%%%%%%%%%%%%%%%%%%%%%%%%%%%%%%%%%%%%%%%%%%%%%%%%%%%%%%%%%
\section*{Tweaks}
\begin{itemize}
\item Compile time generation of static graphs
\item Optimization of the graph to reduce overhead
\item Thread safe implementation of the gvon collectives
\end{itemize}

\cleardoublepage

%%% Local Variables:
%%% TeX-master: "diplom"
%%% End:
