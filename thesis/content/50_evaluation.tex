\chapter{Evaluation}
\label{sec:evaluation}

% Zu jeder Arbeit in unserem Bereich gehört eine Leistungsbewertung. Aus
% diesem Kapitel sollte hervorgehen, welche Methoden angewandt worden,
% die Leistungsfähigkeit zu bewerten und welche Ergebnisse dabei erzielt
% wurden. Wichtig ist es, dem Leser nicht nur ein paar Zahlen
% hinzustellen, sondern auch eine Diskussion der Ergebnisse
% vorzunehmen. Es wird empfohlen zunächst die eigenen Erwartungen
% bezüglich der Ergebnisse zu erläutern und anschließend eventuell
% festgestellte Abweichungen zu erklären.


\begin{itemize}

\item What should be evaluated
  \begin{itemize}
  \item Performance evaluation
  \item Design evaluation (lines of code, reusability)
  \item Flexibility evaluation
  \end{itemize}

\item On which System the implementation was tested
  \begin{itemize}
  \item hypnos with Linux Ubuntu 12.04.4 LTS
  \item Laser queue
    \begin{itemize}
    \item 4 sockets x 8 core AMD Opteron = 32 (+ 2 threads pro core)
    \item 256 GB main memory
    \end{itemize}
  \item k20 queue
    \begin{itemize}
    \item 2 sockets x 4 core Xeon
    \end{itemize}
  \item Network: Infiniband and Ethernet
  \item Communication library
  \end{itemize}

\item What was implemented
  \begin{itemize}
  \item Objective is not to implement efficient simulation, but
    it should demonstrate the developed system in its functionality
    and in comparison with MPI.
    
  \item Simple Game of Life
    \begin{itemize}
    \item Single cell per vertex design
    \item Peer hosts several vertices
    \item Common GoL rules
    \item Next neighbor communication in a 2D Grid with diagonal connections
    \item Just a simple example
    \item But cells can be even more complex
    \end{itemize}

  \item Simple Game of Life MPI 
    \begin{itemize}
    \item Each cell ny one MPI process
    \end{itemize}

  \item Simple N-Body
    \begin{itemize}
    \item http://physics.princeton.edu/~fpretori/Nbody/intro.htm
    \item direct gravitational n-body simulations
    \item compare to mpi n-body simulation
    \end{itemize}

  \item Simple N-Body MPI
    \begin{itemize}
      \item Each body by one MPI process
    \end{itemize}
  \end{itemize}

\item Own expectations
  \begin{itemize}
  \item Not much overhead compared to MPI implementation
  \item Expect the usual communication behavior
  \item less communication over network means less delay
  \item the more local calculation the more performance
  \item The software should perform as fast as the
    underlying adapter
  \end{itemize}

\item Measurements
  \begin{itemize}
  \item Configuration of the developed system (CAL, GRAPH, GVON)
  \item simulations steps per minute
  \item Idea Axel: Delay calculation of some cells
    Remap peers such that execution time of all peers
    will be the same again.
  \item Varying cluster node konfigurations
    \begin{itemize}
      \item Use different amounts of nodes
      \item Use different cluster queues
      \item Use different mapping methods
      \item Evaluate the methods and whether they
        behave like expected.
    \end{itemize}
  \end{itemize}

\item Conclusion about evaluation points. What can
  be done better, where can the system be changed.
  Give reason for behavior of the system.

\end{itemize}

\cleardoublepage

%%% Local Variables:
%%% TeX-master: "diplom"
%%% End:
