\chapter{Introduction}
\label{sec:intro}

% Die Einleitung schreibt man zuletzt, wenn die Arbeit im Großen und
% Ganzen schon fertig ist. (Wenn man mit der Einleitung beginnt - ein
% häufiger Fehler - braucht man viel länger und wirft sie später doch
% wieder weg). Sie hat als wesentliche Aufgabe, den Kontext für die
% unterschiedlichen Klassen von Lesern herzustellen. Man muß hier die
% Leser für sich gewinnen. Das Problem, mit dem sich die Arbeit befaßt,
% sollte am Ende wenigsten in Grundzügen klar sein und dem Leser
% interessant erscheinen. Das Kapitel schließt mit einer Übersicht über
% den Rest der Arbeit. Meist braucht man mindestens 4 Seiten dafür, mehr
% als 10 Seiten liest keiner.



\section{Introduction}

% State of the art petascale systems
The domain of high performance computing is subject to constant
development.  The main focus of this development is the increase of
computing power.  More computing power enables to solve more complex
problems or increased problem sizes.  Current state of the art
computing systems reached the area of petaflop/s (PFLOPS) in the year
2008. That are $10^{15}$ floating point operations per second.  Only a
small number of applications worldwide have taken advantage of so
called petascale systems.

One application, that has shown scalability on petascale systems, is
PIConGPU. It is a particle in cell physic simulation that was running
on the Titan supercomputer at the Oak Ridge National Laboratory in the
United States. By making use of 18.000 Graphic Processing Units (GPUs)
on the Titan, PIConGPU reached 7.1 PFLOPS peak performance.
Nevertheless, applications capable of petascale system are looking
towards systems with more computational power.

Thus, Current computing systems will no longer adequately cover the
future needs of complex and the compute intensive scientific and
industrial applications.  To make more computing power available for
this applications, a rapid progress in the development of computers
and algorithms is necessary.  Next milestone in the development of
supercomputers is the foray of computing systems into an area of an
exaflop/s (EFLOPS) computing power - an exascale system. That are
computing systems that are capable of at least $10^{18}$ floating
point operations per second, a thousandfold increase over the first
petascale supercomputer.  Not all questions for the construction of
such a system are solved. But, it is certain, that exascale will
increase the amount and the complexity of computers to a new
level. Coming up with the challange of reliability, programmability
and usability of such systems. The first exascale systems are expected
to be finished in 2018.

Considering an exascale system, it is not sufficient to simply port
the application that were previously running on a petascale
system. The solely massivly increase in size of the compute system
will lead to an decrease in the mean time to failure and increased
importance of data locality.  Furthermore, emergence of accelerator
hardware make computing system more heterogeneous and hierarchical and
more complex to program and use. Therefore, the high performance
application has to be adapted onto upcoming circumstances of exascale
systems. It exists the need for a smart description concept for high
performance applications, that fits on upcoming heterogeneous,
hierarchical, exascale computing systems.

% Why is communication necessary
\begin{itemize}
\item Since, communication is an often used approach to interconnect
  parallel entities of an application.
\item One possible approach to scale on upcoming systems is to enhance
  well known communication libraries.
\item Existing communication will be taken as fundament
\item On top of this fundament additional layers will be installed to
  improve flexibility of the communication library
\end{itemize}

% How such a system should be
\begin{itemize}
\item It were designed and implemented layers on top of existing
  communication libraries that map common communication processes onto
  these libraries.

\item Abstraction from existing communication libraries
\item Modeling of the communication topology of the simulation
  application
\item Mapping from the communication topology onto the communication
  abstraction
\end{itemize}

% Overview following section
\begin{itemize}
\item The following section \ref{sec:technical_background} is a
  summary of technical background, giving a brief overview of the
  basic terms in high performance computing and communication in
  computing systems.

\item Section \ref{sec:related_work} examines research in the area of
  interest.
\item Versatile readers can skip this section and continue straight
  with the design section \ref{sec:design}.
\item There will be discussed and justified design decisions.
\item An implementation for the designed system is presented in
  section \ref{sec:impl}.
\item Finally, evaluation of the prototype in contrast to classic MPI
  implementation


\end{itemize}


\cleardoublepage

%%% Local Variables:
%%% TeX-master: "diplom"
%%% End:
