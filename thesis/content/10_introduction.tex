\chapter{Introduction}
\label{sec:intro}

% Die Einleitung schreibt man zuletzt, wenn die Arbeit im Großen und
% Ganzen schon fertig ist. (Wenn man mit der Einleitung beginnt - ein
% häufiger Fehler - braucht man viel länger und wirft sie später doch
% wieder weg). Sie hat als wesentliche Aufgabe, den Kontext für die
% unterschiedlichen Klassen von Lesern herzustellen. Man muß hier die
% Leser für sich gewinnen. Das Problem, mit dem sich die Arbeit befaßt,
% sollte am Ende wenigsten in Grundzügen klar sein und dem Leser
% interessant erscheinen. Das Kapitel schließt mit einer Übersicht über
% den Rest der Arbeit. Meist braucht man mindestens 4 Seiten dafür, mehr
% als 10 Seiten liest keiner.


\section{Introduction}

Only a few problems can be calculated without exchanging information
between other nodes. Thus each node has the challange to retrieve all
the data, that is needed to calculate the next step of its
computation.

\begin{itemize}
\item Computational power of single computer are limited by phsical
  laws
\item Single computers have only a limited amount of memory
\item Applications / Simulations need to be decomposed into subdomains
\item Subdomains need to be computed on varying computers
\item Computers of a cluster need to work together
\item Work together means, exchanging data between computers
\item Communication plays an important role in cluster systems
\item Cluster systems are the future of high performance computing for complex simulations
\item Cluster systems are extendable and scale with the simulation complexity
\end{itemize}


\cleardoublepage

%%% Local Variables:
%%% TeX-master: "diplom"
%%% End:
