\chapter{Introduction}
\label{sec:intro}

% Die Einleitung schreibt man zuletzt, wenn die Arbeit im Großen und
% Ganzen schon fertig ist. (Wenn man mit der Einleitung beginnt - ein
% häufiger Fehler - braucht man viel länger und wirft sie später doch
% wieder weg). Sie hat als wesentliche Aufgabe, den Kontext für die
% unterschiedlichen Klassen von Lesern herzustellen. Man muß hier die
% Leser für sich gewinnen. Das Problem, mit dem sich die Arbeit befaßt,
% sollte am Ende wenigsten in Grundzügen klar sein und dem Leser
% interessant erscheinen. Das Kapitel schließt mit einer Übersicht über
% den Rest der Arbeit. Meist braucht man mindestens 4 Seiten dafür, mehr
% als 10 Seiten liest keiner.



\section{Introduction}

\begin{itemize}

% State of the art petascale systems
\item Current state of the art computing systems operate in area of
  petaflop/s
\item Only a small number worldwide have take advantage of a
  petaflop/s system
\item PIConGPU (a physic simulation) is a physic application that has
  shown the ability for execution on Titan supercomputer at the Oak
  Ridge National Laboratory in the United States
\item The Titan is a supercomputer with about 27 petaflop/s peak
  performance.
\item Nevertheless, these applications are looking towards systems
  with more computational power

% Exascale systems
\item Rapid progress in the development of computers and algorithms
\item Symbiotic evolution of computing systems and high performance
  applications
\item Thus, the more compute power is necessary and soon available,
  since current computers can no longer adequately cover the future
  needs of complex and the compute intensive scientific and industrial
  applications.
\item Next milestone in the development of supercomputers, foray of
  computing systems into with an exaflop/s computing power - an
  exascale system
\item That are computing systems that are capable of at least
  $10^{18}$ flop/s
\item That is thousandfold increase over the first petascale computer
  from 2008.
\item Not all questions of the construction of such a system are
  solved
\item But, it is certain, that exascale will increasing the amount and
  the complexity of computers
\item It is expected that the construction of the first exascale
  systems will be finished in 2018.

% Upcoming problems with exascale system
\item Applications with the objective to scale onto exascale systems
  have to be prepared
\item Challenge of reliability, programmability, power consumption and
  usability
\item Increasing failure probability of computer hardware
\item Furthermore, emergence of accelerator hardware make computing
  system more heterogeneous and hierarchical and more complex to
  program and use.
\item It is not enough to just increase computing power
\item Therefore, simulation application has to be adapted onto
  upcoming circumstances of exascale systems.
\item Need for smart description concept that fit on future
  heterogeneous, hierarchical, exascale computing systems.
\item Foundation for load distribution and fault tolerant systems

% Developed system
\item Foundation for such a system is a portable and flexible
  communication approach
\item portable, so that deployment on varying computing systems is
  possible
\item ``For portability reasons, most of the parallel applications are on top
  of widespread message passing interfaces such as PVM or MPI. But
  this restricts the application to a single compute environment and
  make an interconction of varying compute environments very hard.''
\item flexible, so that mapping of communication processes can be
  adjusted onto the needs of the simulation and the computing system.
\item This behavior is achieved by abstraction from existing
  communication libraries, modeling of the communication topology of
  the arbitrary simulation and a mapping from the communication
  topology onto the communication abstraction.

% Overview following section
\item The following section \ref{sec:technical_background} is a
  summary of technical background, giving a brief overview of the
  basic terms in high performance computing and communication in
  computing systems.
\item Section \ref{sec:related_work} examines research in the
  area of interest.
\item first design of such an system
\item implementation of an prototype
\item evaluation of the prototype in contrast to classic MPI implementation

\end{itemize}


\cleardoublepage

%%% Local Variables:
%%% TeX-master: "diplom"
%%% End:
