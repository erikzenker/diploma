\chapter{Current State of the Art}
\label{sec:state}

% Hier werden zwei wesentliche Aufgaben erledigt:

% 1. Der Leser muß alles beigebracht bekommen, was er zum Verständnis
% der späteren Kapitel braucht. Insbesondere sind in unserem Fach die
% Systemvoraussetzungen zu klären, die man später benutzt. Zulässig ist
% auch, daß man hier auf Tutorials oder Ähnliches verweist, die hier auf
% dem Netz zugänglich sind.

% 2. Es muß klar werden, was anderswo zu diesem Problem gearbeitet
% wird. Insbesondere sollen natürlich die Lücken der anderen klar
% werden. Warum ist die eigene Arbeit, der eigene Ansatz wichtig, um
% hier den Stand der Technik weiterzubringen? Dieses Kapitel wird von
% vielen Lesern übergangen (nicht aber vom Gutachter ;-), auch später
% bei Veröffentlichungen ist "Related Work" eine wichtige Sache.

% Viele Leser stellen dann später fest, daß sie einige der Grundlagen
% doch brauchen und blättern zurück. Deshalb ist es gut,
% Rückwärtsverweise in späteren Kapiteln zu haben, und zwar so, daß man
% die Abschnitte, auf die verwiesen wird, auch für sich lesen
% kann. Diese Kapitel kann relativ lang werden, je größer der Kontext
% der Arbeit, desto länger. Es lohnt sich auch! Den Text kann man unter
% Umständen wiederverwenden, indem man ihn als "Tutorial" zu einem
% Gebiet auch dem Netz zugänglich macht.

% Dadurch gewinnt man manchmal wertvolle Hinweise von Kollegen. Dieses
% Kapitel wird in der Regel zuerst geschrieben und ist das Einfachste
% (oder das Schwerste weil erste).


This chapter provides first of all an introduction into technical
background. Readers that are well versed in the are of high performance
computing and communication in cluster systems can skip this part of
the chapter. Secondly, it provides a brief overview about
related work.

%%%%%%%%%%%%%%%%%%%%%%%%%%%%%%%%%%%%%%%%%%%%%%%%%%%%%%%%%%%%%%%%%%%%%%%%%%%%%%%%
%                                                                              %
% TECHNICAL BACKGROUND                                                         %
%                                                                              %
%%%%%%%%%%%%%%%%%%%%%%%%%%%%%%%%%%%%%%%%%%%%%%%%%%%%%%%%%%%%%%%%%%%%%%%%%%%%%%%%
% Checked
\section{Technical Background}
\label{sec:technical_background}
This section gives all necessary basics for understanding of the
development of the system.  These basics are the foundations of
cluster computing in Section~\ref{sec:cluster}, an introduction into
many-core architectures and accelerator devices in
Section~\ref{sec:accel} and a brief overview on domain decomposition
of applications in
Section~\ref{sec:domain_decomposition}. Furthermore, communication
techniques in cluster systems are discussed in
Section~\ref{sec:communication} and it follows a definition of the
term jungle computing in Section~\ref{sec:jungle}. Finally, the
simulation application PIConGPU is introduced in
Section~\ref{sec:picongpu}.

%%%%%%%%%%%%%%%%%%%%%%%%%%%%%%%%%%%%%%%%%%%%%%%%%%%%%%%%%%%%%%%%%%%%%%%%%%%%%%%%
%                                                                              %
% HIGH PERFORMANCE COMPUTING                                                   %
%                                                                              %
%%%%%%%%%%%%%%%%%%%%%%%%%%%%%%%%%%%%%%%%%%%%%%%%%%%%%%%%%%%%%%%%%%%%%%%%%%%%%%%%
% Checked
\subsection{Cluster Computing}
\label{sec:cluster}
Parallel computers are not anymore specialized traditional
supercomputing platforms, instead low cost off-the-shelf commodity
computers form so called compute clusters \cite{ref:hpcc1}.

A compute cluster is in principle a distributed memory architecture
and forms a homogeneous network of similar compute entities, which
will be called nodes. Each node is equipped with one or more central
processing units (CPUs) and with a growing popularity also with one or
more accelerator devices (e.g. GPUs) \cite{ref:accel}. Accelerator
devices will be discussed in Section~\ref{sec:accel}

The main memory between the nodes is not shared, instead every node
has its own local memory.  To exchange data between the local memories
of the nodes, the nodes are interconnected via a high-speed network
connection with small-latency and high-bandwidth,

The calculation power of a high performance computing cluster is used
where time to solution is important and the memory of a single
computer does not cover the demand of an application. These are
physical \cite{ref:picongpu}, weather, astronomy, multimedia, medical
imaging and biobanking simulations.


%%%%%%%%%%%%%%%%%%%%%%%%%%%%%%%%%%%%%%%%%%%%%%%%%%%%%%%%%%%%%%%%%%%%%%%%%%%%%%%%
%                                                                              %
% MANY-CORE ARCHITECTURES AND ACCELERATORS                                     %
%                                                                              %
%%%%%%%%%%%%%%%%%%%%%%%%%%%%%%%%%%%%%%%%%%%%%%%%%%%%%%%%%%%%%%%%%%%%%%%%%%%%%%%%
% Checked
\subsection{Many-Core Architectures and Accelerators}
\label{sec:accel}

Traditional cluster, grid, and cloud computing systems are adding more
and more state-of-the-art many-core architectures or accelerator
devices to their compute nodes to exploit high peak performance for
their applications. This changes clusters from homogeneous systems,
where a node has only one type of hardware, to heterogeneous systems
with varying types of hardware.

An accelerator device is some kind of a coprocessor which is usually
able to execute operations in a massively parallel manner.  The
devices are commonly optimized for a handful of operations (e.g
floating point operations) in which they can outperform traditional
CPU hardware. But, these devices do also require special treatment in
regard to programming.

Particularly, General-purpose computing on graphics processing units
(GPGPU), mainly manufactured from AMD or Nvidia, enter the world of
compute clusters. Nvidia has pushed forward CUDA\cite{ref:cuda} as the
preferred model and language to program their
GPUs. OpenCL\cite{ref:opencl}, as alternative to CUDA, adopts a more
general approach.  It provides a common model for accelerator
programming and covers a wide range of accelerator devices.  Another
member of the accelerator device family is the Xeon
Phi\cite{ref:xeon_phi} from Intel. The Xeon Phi is a many-core
coprocessor based on interconnected legacy x86 cores on a single
chip. While, Xeon Phis can be programmed by OpenCL, Intel suggest to
exploit the device performance by OpenMP \cite{ref:openmp}.

Usually, the accelerator devices need a CPU to manage their memory
transfers between main memory and device memory and also for internode
communication between the nodes. In this so called offload mode the
CPU is called the host of the accelerator device.

But, accelerator devices have the future objective to run in a
standalone or native mode, where a host is not needed anymore.  Which
implies, that these devices are connected directly to a network,
giving them the possibility to communicate directly with each
other. The Xeon Phi is already able to run in native mode and Nvidia
also tries to get rid of the host CPU with GPUDirect.

Supercomputers of the TOP 500 list \cite{ref:top500} are already
populated by accelerator devices, which demonstrated the development
of compute clusters to heterogeneous cluster systems.


%%%%%%%%%%%%%%%%%%%%%%%%%%%%%%%%%%%%%%%%%%%%%%%%%%%%%%%%%%%%%%%%%%%%%%%%%%%%%%%%
%                                                                              %
% DOMAIN DECOMPOSITION                                                         %
%                                                                              %
%%%%%%%%%%%%%%%%%%%%%%%%%%%%%%%%%%%%%%%%%%%%%%%%%%%%%%%%%%%%%%%%%%%%%%%%%%%%%%%%
% Checked
\subsection{Domain decomposition}
\label{sec:domain_decomposition}
Domain decomposition is a very popular method to separate the
computational domain of an application in smaller chunks of work,
subdomains, which are mostly solveable independent from each
other. This gives an application the chance to exploit the tremendous
performance gain of parallel computers.

We usually do not live in a perfect parallel world, where each
subdomain is totally independent from all other subdomains.
Therefore, communication or coordination between certain subdomains is
necessary. How communication is covered in an cluster environment
is covered in Section~\ref{sec:communication}.

The Decomposition of the application domain can be partitioned into three
levels: subdomains which make local computations, the
communication with neighboring subdomains and the global communication
operations which include concurrent communication of several or all
subdomains.

The challenge of domain decomposition methods is to map these three
levels efficiently onto a parallel computer. So that, the application
exploits the available computing resources.


%%%%%%%%%%%%%%%%%%%%%%%%%%%%%%%%%%%%%%%%%%%%%%%%%%%%%%%%%%%%%%%%%%%%%%%%%%%%%%%%
%                                                                              %
% TRADITIONAL COMMUNICATION MECHANISMS IN CLUSTER SYSTEMS                      %
%                                                                              %
%%%%%%%%%%%%%%%%%%%%%%%%%%%%%%%%%%%%%%%%%%%%%%%%%%%%%%%%%%%%%%%%%%%%%%%%%%%%%%%%
% Checked
\subsection{Traditional Communication Mechanisms in Cluster Systems}
\label{sec:communication}
Communication has a central role in complex simulation
applications. The calculation of large data sets on a single computing
device in a acceptable amount of time is not possible anymore.
Therefore, the simulation need to be decomposed in subdomains and
distributed to more than one computing device. Subdomains are usually
not independent from each other, they need to exchange data.

The most simple way for data exchange is the direct communication of
nodes (peer-to-peer).  This can be done on protocols such as TCP, UDP,
IP using Berkeley sockets (\ref{sec:tcp_udp_ip}), which is
communication in the UNIX way; everything is a file or file
descriptor. It provides a classical send-receive model of messages or
data.  But Sockets where not designed for the usage in a high
performance computing environment in mind. Libraries that are built
directly on top of sockets are Boost Asio \cite{ref:boost_asio} and
ZMQ \cite{ref:ZMQ}.

Active messages \cite{ref:am} is a slight departure from the classical
send-receive model. It is based on one-sided communication operations,
which transfer messages regardless of the current activity of the
receiver process. It was designed to reduce the impact of
communication overhead on application performance.

A more modern and abstract model is the Message Passing Interface. It
provides a more abstract interface for communication and is available
on top of several communication libraries. Because parts of MPI will
be used for the implementation of the system, an introduction into MPI
will be provided in Section~\ref{sec:mpi}.

An other way for data exchange, is transformation of a cluster into a
single large parallel computer with a single address space.  The
operating system MOSIX \cite{ref:mosix} is a Linux based
multi-computer OS (MOS), that provides a single-systems
image\cite{ref:single_system_image} as if using one computer with
multiple CPUs. The Parallel Virtual Machine\cite{ref:pvm} is a
software package that permits a heterogeneous collection of Unix
and/or Windows computers hooked together by a network to be used as a
single computer.

These distributed shared architectures make it possible to use
standard and simple parallelization approaches such as Pthreads or
OpenMP.  Communication between nodes is covered by shared memory and
does not have to be modeled explicitly.  Therefore, it is easy to ship
existing parallel applications for a single machine onto these kind of
systems. But with higher abstraction always comes a loose of control
and also efficiency. The underlying compute cluster has to provide
such an architecture, which limits the application to this kind of
clusters.

%%%%%%%%%%%%%%%%%%%%%%%%%%%%%%%%%%%%%%%%%%%%%%%%%%%%%%%%%%%%%%%%%%%%%%%%%%%%%%%%
%                                                                              %
% MPI                                                                          %
%                                                                              %
%%%%%%%%%%%%%%%%%%%%%%%%%%%%%%%%%%%%%%%%%%%%%%%%%%%%%%%%%%%%%%%%%%%%%%%%%%%%%%%%
% Checked
\subsubsection{Message passing intferface (MPI)}
\label{sec:mpi}
The Message Passing Interface (MPI) is a standardized and portable
message-passing specification, specified by the Message Passing
Interface Forum \cite{ref:mpi_specification}. It has become the
defacto standard for communication in cluster systems.

The specification is language independent and its implementations are
available on virtually every parallel computer system.  Popular
implementations are MPICH and OpenMPI.  The MPI implementations are
implemented on top of physical networks provided by the underlying
cluster system ( AM, Ethernet, Infiniband, Myrinet etc.)

The first version of the specification, known as MPI-1, was completed
in 1994 making use of the most attractive features of already
existing message passing systems.  Further improvements followed 1997
by version MPI-2 and 2008 by version MPI-3. Version numbers follow
the policy that major versions like MPI-2 or MPI-3 come with new
features to the MPI standard and minor versions like MPI-1.1 or
MPI-1.2 come with clarifications of already existing features. In
the following the key features of the varying MPI versions are listed:

\begin{itemize}
  \item MPI-1 (1994)
    \begin{itemize}
      \item Basic point-to-point communication
      \item Collectives
      \item Derived data types
    \end{itemize}
  \item MPI-2 (1997)
    \begin{itemize}
      \item Parallel I/O
      \item Remote memory operations (one-sided)
      \item Dynamic process management
    \end{itemize}
  \item MPI-3 (2008)
    \begin{itemize}
      \item Non-blocking collectives
      \item Improved one-sided communication interface
    \end{itemize}
\end{itemize}

An MPI program is a set of autonomous processes, executing their own
code, in an MIMD style \cite{Flynn:1972:COE:1952456.1952459}. Each
process rests in his own address space and communicates via MPI
communication primitives with other processes in the same MPI
program.

Processes are identified according to their relative rank in a group
by consecutive integers from 0 to the group size minus 1. The
initial group compromises the full set of processes with ranks in the
range 0 to number of processes minus one. A typical MPI program run is
composed from the following steps:

\begin{enumerate}
\item Deploy MPI program onto the computing nodes
\item Start MPI program
\item Initialize MPI environment
\item Each process retrieves own rank
\item Each process retrieves number of ranks overall
\item Depending on own rank and number of overall processes do some task\\
  (varying flow of control between ranks)
\item Exchange data with other processes
\item Finalize MPI environment
\item Exit MPI program
\end{enumerate}

MPI provides a large and versatile set of routines, that fit the needs
of the parallel programming community. At its most basic, MPI has
functions for sending and receiving a message (point-to-point
operations).  These functions are available both blocking and
non-blocking. While blocking, interrupts the program flow until the
operation has finished, non-blocking operations are performed in the
background.

Furthermore, a variety of collective operations, that involve a group
of processes for communication, are defined. Collective operations
collect, distribute and reduce data of the MPI processes.  All members
of the group have to participate to successfully execute the
collective operation.  The result of the operation is either received
by one process, the so called root, or by all processes of a group.  A
collective operation can only be successful, when all peers of a
context are participating. If only one peer does not execute the
operation then the application is stuck. It will only continue when
this peer also executes this collective operation.


%%%%%%%%%%%%%%%%%%%%%%%%%%%%%%%%%%%%%%%%%%%%%%%%%%%%%%%%%%%%%%%%%%%%%%%%%%%%%%%%
%                                                                              %
% JUNGLE COMPUTING                                                             %
%                                                                              %
%%%%%%%%%%%%%%%%%%%%%%%%%%%%%%%%%%%%%%%%%%%%%%%%%%%%%%%%%%%%%%%%%%%%%%%%%%%%%%%%
\subsection{Jungle computing}
\label{sec:jungle}
Jungle computing can be seen as a further evolution step after grid
computing. The initial intention of grid computing was the request to
bundle the tremendous compute power of cluster systems over the
internet\cite{ref:grid}. It was meant to an easy to use system over a
set of distributed resources, which delivers potentially efficient
computing, even at a worldwide scale.  The hook with grid computing
is, that every compute systems has its own administration environment,
submit system, communication middleware, and programming interface.

Even though, every system viewed in isolation, by using platform
specific code, designing applications that interconnects several
compute systems can be a hard challenge.  With the increasing desire
for speed, scalability and flexibility in many scientific research
domains, it exist a demand for easy end-user access to multiple and
potentially very diverse platforms.

``A true computing jungle'' where the words by which F. J. Seinastra
introduced such heterogeneous systems as jungle computing
systems\cite{ref:jungle}. It is defined as simultaneously using a
combination of heterogeneous, hierarchical and distributed computing
resources.

This definition describes the undergoing revolutionary change in the
high performance computing (HPC) area very well. On one hand,
computations will be distributed to several compute clusters and on
the other hand the underlying hardware of compute nodes are changing
to heterogeneous hardware in the years to come. The efficient
programming by scientist of such systems, and thereby the efficient
mapping of computational problems onto the jungle has become
enormously difficult to maintain and overview.


%%%%%%%%%%%%%%%%%%%%%%%%%%%%%%%%%%%%%%%%%%%%%%%%%%%%%%%%%%%%%%%%%%%%%%%%%%%%%%%%
%                                                                              %
% PIConGPU - THE APPLICATION                                                   %
%                                                                              %
%%%%%%%%%%%%%%%%%%%%%%%%%%%%%%%%%%%%%%%%%%%%%%%%%%%%%%%%%%%%%%%%%%%%%%%%%%%%%%%%
% Checked
\subsection{PIConGPU - the application}
\label{sec:picongpu}
PIConGPU is a relativistic particle-in-cell (PIC) code designed for
simulating laser-plasma interactions \cite{ref:picongpu}.  The
application is optimized for the execution on cluster systems with
NVIDIA GPU-accelerated nodes (Section~\ref{sec:accel}), whereby the
host CPUs are only utilized for memory transfers between CPU and GPU
and for internode communication via MPI (Section~\ref{sec:mpi}). It
was demonstrated that PIConGPU is capable to scale on systems with up
to 18.000 GPUs \cite{ref:picongpu_scale}.

The simulation domain, a volume, is decomposed into subdomains
\ref{sec:domain_decomposition} forming a three-dimensional grid. Each
grid cell is mapped onto a GPU. Particles in a cell interact on one
hand with particles in other cells and on the other hand a cross the
border of the cells during the simulation. Furthermore, fields of a
cell interact with fields of neighboring cells. Thus, GPUs need to
exchange data among cells borders. This data is not transferred
directly between the GPUs hosting the cells, it has to take the detour
via host and network.

%%%%%%%%%%%%%%%%%%%%%%%%%%%%%%%%%%%%%%%%%%%%%%%%%%%%%%%%%%%%%%%%%%%%%%%%%%%%%%%%
%                                                                              %
% RELATED WORK                                                                 %
%                                                                              %
%%%%%%%%%%%%%%%%%%%%%%%%%%%%%%%%%%%%%%%%%%%%%%%%%%%%%%%%%%%%%%%%%%%%%%%%%%%%%%%%
\section{Related Work}
\label{sec:related_work}
The developed system is related to systems that make it possible to
decompose an application into smaller chunks of work and describe the
relationship between this chunks.  Thus, there is some relationship to
workflow and many task computing systems. These systems divide the
application in tasks and perform a resource mapping or scheduling of
tasks onto the available computing resources. The scheduling algorithm
considers thereby the dependencies among tasks. Some systems allow a
dynamic rescheduling of tasks at run-time for a better compute load
distribution.

% MPI Topology
% Checked
\paragraph*{MPI Topology}
An approach directly embedded in MPI, that describes dependencies of
processes, are MPI virtual topologies \cite{ref:mpi_topology}. An MPI
topology describes the communication pattern of a set of MPI processes
in a group through a cartesian grid or a graph.  This topology is
independent from the physical network structure and therefore the
mapping from MPI processes onto the hardware topology can be optimized
by MPI process reordering. Thereby, is the number of vertices of the
graph less or equal to the number of MPI processes, so that a
one-to-one mapping is possible.

Boost MPI provides the possibility to describe the process topology by
then Boost Graph Library (BGL)\cite{ref:boost_bgl}.  The resulting
Graph Communicator \cite{ref:boost_graph_communicator} gives the
possibility to communicate based on the graph topology.

In summary, MPI topology uses the well established mechanism to
decompose the application in MPI processes, models their relationship
and uses this information for an optimized deployment of MPI processes.
Since this approach is statically based on MPI, the communication
library is not exchangable even if MPI already supports a wide range
of communication libraries.

% ULL_calibrate_lib
\paragraph*{ULL\_calibrate\_lib}
A library focusing on the balancing load among cluster systems
supporting MPI is the ULL\_calibrate\_lib
\cite{ref:ull_calibrate_lib}. It allows dynamic load balancing by
enclosing critical section of the source code by library methods
calls. The library compares the run-time of these enclosed tasks. Does
the task exceeds a predefined threshold, the library tries to
load balance these tasks. The load balancing and communication is based on
the MPI platform.

The library uses communication mechanism from MPI as foundation and
enhances this MPI with load balancing features. Thus, this library
provides process redistribution features on MPI level. But, since it
is fixed to the utilization of MPI, it offers also the same MPI
problems.

% Pegasus
% Checked
\paragraph*{Pegasus}
The Pegasus system \cite{ref:pegasus} decomposes an application into
tasks and models task dependencies by an abstract workflow
representation. Workflows are modeled explicit as directed acyclic
graphs representing the composition of task and data dependencies. The
creation of application specific workflows is supported by tools like
Chimera \cite{ref:chimera} and the Composition Analysis Tool (CAT)
\cite{ref:cat}.  Tasks are submitted to the queuing system, that
schedules these tasks onto the available compute resources.

Pegasus do not model explicit communication processes among tasks.
Therefore, communication topologies are hard to describe by the used
workflow system approach.


% StarPU
% Checked
\paragraph*{StarPU} 
A similar approach such as Pegasus is applied by the StarPU system
\cite{ref:starpu}.  It is a unified execution model for tasks, that
manages the execution of parallel tasks on heterogeneous hardware.
Each task is implemented by a platform specific code, and the system
takes care about task offloading and load balancing. Each target
platform has to provide a driver that provides methods for memory
transfer to the target platform and task execution. StarPU already
implements drivers for multicore CPUs, the CUDA platform and the CELL
processor.

% Charm++
% Checked
\paragraph*{Charm++} 
An approach wich embeds into the C++ programming language is
Charm++ \cite{ref:charm}. Charm++ has the concept of describing the
application through so called charmes which describe data and work
units. The units are over-decomposed. Thus, multiple work and data
units are assinged to each processing element. They are not strictly
bounded to processing elements. Instead, they can be migrated onto
each processing element during execution. Charmes interact among each
other with asynchronous method innvocation.

The downside of Charm++ is that an application has be transformed into
a set of charmes, which may change the principle mechanics of an
application.

% Ibis
% Checked
\paragraph*{Ibis} 
Ibis \cite{ref:ibis} provides two logically independent subsystems.
On one hand the programming system and on the other hand the
deployment system. The programming system provides several programming
models (MPI, RMI, Satin) that are implemented on the same
communication library, the Ibis Portability Layer (IPL).  The IPL
provides a range of communication primitives including point-to-point,
multicast, streaming, serialization and deserialization. These
primitives are implemented by adapters, adapting existing
communication libraries like SmartSockets, TCP UDP, Bluetooth, Myrinet and
MPI.

The deployment system is based on the Java Grid Application Toolkit
(JavaGAT).  It has the philosophy, that the development and
compilation is done locally on a workstation and is then deployed from
there onto the distributed computing system.

The Ibis platform is based on Java technologies. All subsystem come as
jar file and do not need any library installation. It is easily
applicable for application written in java. Non-java applications can
make use of ibis through JNI, but it is complicated.

% Constellation
% Checked
\paragraph*{Constellation}
A software platform for distributed, heterogeneous and hierarchical
computing is the Constellation \cite{ref:constellation} framework. An
application is modeled by distinct activities and each activity
represents a distinct action on the application.  Activities can be
related among each other by dependencies and are labeled with tags
that define execution conditions like target system and data locality.
Activities are implemented independently for their specific target
system (MPI, CUDA, OpenMP) and are then scheduled to execution units
on the target system. This mapping onto the available executors is
performed automatically with respect to tags and
heterogeneity. Constellation is implemented in Java on top of Ibis,
but is not restricted to Ibis.

\paragraph*{}
The approaches presented by Ibis and Constellation meet very close the
idea of a platform for computations in a heterogeneous distributed
environment. Both, are implemented in Java and therefore do not
satisfy the requirements for integration in high performance computing
application. But, Ibis and Constellation point the way for the
development of the system.


\cleardoublepage

%%% Local Variables:
%%% TeX-master: "diplom"
%%% End:
