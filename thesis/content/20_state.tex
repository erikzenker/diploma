\chapter{Technical Background}
\label{sec:state}

% Hier werden zwei wesentliche Aufgaben erledigt:

% 1. Der Leser muß alles beigebracht bekommen, was er zum Verständnis
% der späteren Kapitel braucht. Insbesondere sind in unserem Fach die
% Systemvoraussetzungen zu klären, die man später benutzt. Zulässig ist
% auch, daß man hier auf Tutorials oder Ähnliches verweist, die hier auf
% dem Netz zugänglich sind.

% 2. Es muß klar werden, was anderswo zu diesem Problem gearbeitet
% wird. Insbesondere sollen natürlich die Lücken der anderen klar
% werden. Warum ist die eigene Arbeit, der eigene Ansatz wichtig, um
% hier den Stand der Technik weiterzubringen? Dieses Kapitel wird von
% vielen Lesern übergangen (nicht aber vom Gutachter ;-), auch später
% bei Veröffentlichungen ist "Related Work" eine wichtige Sache.

% Viele Leser stellen dann später fest, daß sie einige der Grundlagen
% doch brauchen und blättern zurück. Deshalb ist es gut,
% Rückwärtsverweise in späteren Kapiteln zu haben, und zwar so, daß man
% die Abschnitte, auf die verwiesen wird, auch für sich lesen
% kann. Diese Kapitel kann relativ lang werden, je größer der Kontext
% der Arbeit, desto länger. Es lohnt sich auch! Den Text kann man unter
% Umständen wiederverwenden, indem man ihn als "Tutorial" zu einem
% Gebiet auch dem Netz zugänglich macht.

% Dadurch gewinnt man manchmal wertvolle Hinweise von Kollegen. Dieses
% Kapitel wird in der Regel zuerst geschrieben und ist das Einfachste
% (oder das Schwerste weil erste).

%%%%%%%%%%%%%%%%%%%%%%%%%%%%%%%%%%%%%%%%%%%%%%%%%%%%%%%%%%%%%%%%%%%%%%%%%%%%%%%%
%                                                                              %
% TECHNICAL BACKGROUND                                                         %
%                                                                              %
%%%%%%%%%%%%%%%%%%%%%%%%%%%%%%%%%%%%%%%%%%%%%%%%%%%%%%%%%%%%%%%%%%%%%%%%%%%%%%%%



%%%%%%%%%%%%%%%%%%%%%%%%%%%%%%%%%%%%%%%%%%%%%%%%%%%%%%%%%%%%%%%%%%%%%%%%%%%%%%%%
%                                                                              %
% HIGH PERFORMANCE COMPUTING                                                   %
%                                                                              %
%%%%%%%%%%%%%%%%%%%%%%%%%%%%%%%%%%%%%%%%%%%%%%%%%%%%%%%%%%%%%%%%%%%%%%%%%%%%%%%%
\section{Cluster computing}
\label{sec:cluster}
Parallel computers are not anymore specialized traditional
supercomputing platforms, instead low cost off-the-shelf commodity
computers form so called compute clusters \cite{ref:hpcc1}.

A compute cluster is in principle a distributed memory architecture
and forms a homogeneous network of individual compute entities, which
will be called nodes. Each node is equipped with one or more central
processing units (CPUs) and with a growing popularity also with one or
more accelerator devices (e.g. GPUs) \cite{ref:accel}.

The memory between the nodes is not shared, thus usually every node
has its own local memory. To exchange data between the local memorys
of the nodes, the nodes are interconnected via a high-speed network
connection with small-latency and high-bandwidth.

\todo{search for applications of compute clusters}
The calculation power of a high performance computing cluster
is used where time to solution is important and the memory
of a single computer does not cover the demand of an application. These are
physical simulations\cite{ref:picongpu}, weather simulations.

\begin{itemize}
  \item Source: Grids, Clouds and Virtualization Cafaro, Massimo Aloisio, Giovanni
  \item Source: High-Performance Computing on Complex Environments
\end{itemize}

%%%%%%%%%%%%%%%%%%%%%%%%%%%%%%%%%%%%%%%%%%%%%%%%%%%%%%%%%%%%%%%%%%%%%%%%%%%%%%%%
%                                                                              %
% MANY-CORE ARCHITECTURES AND ACCELERATORS                                     %
%                                                                              %
%%%%%%%%%%%%%%%%%%%%%%%%%%%%%%%%%%%%%%%%%%%%%%%%%%%%%%%%%%%%%%%%%%%%%%%%%%%%%%%%
\section{Many-core architectures and accelerators}
\label{sec:accel}

Traditional clusters, grids, and cloud systems are adding more and
more state-of-the-art many-core architectures or accelerator devices
to their compute nodes to exploit high peak performance for their applications.

Accelerator devices are some kind of coprocessor which are usually
able to execute operations in a massively parallel manner.  The
devices are commonly optimized for a handful of operations (e.g
floating point operations) in which they can outperform traditional
CPU hardware. But, this devices do also require special treatment in
regard to programming.

Particularly, General-purpose computing on graphics processing units
(GPGPU), mainly manifactured from AMD or Nvidia, enter the world of
compute clusters. Nvidia has pushed forward CUDA\cite{ref:cuda} as the
prefered model and language to programm their
GPUs. OpenCL\cite{ref:opencl} adopts a more general approach as a
common model for heterogeneous programming for a wide range of
accelerator devices.

Another member of the accelerator device familie is the xeon
phi\cite{ref:xeon_phi} from intel. The xeon phi is a many-core
coprocessor based on interconntected legacy x86 cores on a single chip.

Usually, the accelerator devices need a CPU to manage the memory
transfer between main memory and device memory and also for internode
communication between the nodes. In this so called offload mode the
CPU is called host and the accelerator device just device.

But accelerator devices have the future objective to run in a
standalone or native mode, where a host is not needed anymore. 
Which also implies that these devices are connected directly to
a network connection, what gives them to possibility to communicate
directly with each other. The Xeon Phi is allready able to run in native mode.

Supercomputers of the TOP500 are allready populated by accelerator
devices, which demonstrated the developement of compute clusters to
heterogeneous systems and it also shows that the hpc community should
be able to programm such systems.

%%%%%%%%%%%%%%%%%%%%%%%%%%%%%%%%%%%%%%%%%%%%%%%%%%%%%%%%%%%%%%%%%%%%%%%%%%%%%%%%
%                                                                              %
% DOMAIN DECOMPOSITION                                                         %
%                                                                              %
%%%%%%%%%%%%%%%%%%%%%%%%%%%%%%%%%%%%%%%%%%%%%%%%%%%%%%%%%%%%%%%%%%%%%%%%%%%%%%%%
\section{Domain decomposition}
\label{sec:domain_decomposition}

Domain decomposition is a very popular method to seperate
computational problem in smaller chunks of work, which are solveable
independent from each other. This gives an application the chance to
exploit the tremendous performace gain of parallel computers.

In The simplest form of domain decomposition each subdomain lies on
its own process. Usually we do not live in a perfect parallel world in
such communication or coordinatin between the processes is needed
\ref{sec:communication}.

Decomposition of domains can be seen in three levels: the set of
subdomains which make some local computations, the communication with
neighboring cells and global communication operations which include
concurrent communication of several or all subdomains.

The challange of domain decomposition mehtods is to map these three
levels efficiently onto a parallel computer.

\begin{itemize}
\item 
\end{itemize}


%%%%%%%%%%%%%%%%%%%%%%%%%%%%%%%%%%%%%%%%%%%%%%%%%%%%%%%%%%%%%%%%%%%%%%%%%%%%%%%%
%                                                                              %
% TRADITIONAL COMMUNICATION MECHANISMS IN CLUSTER SYSTEMS                      %
%                                                                              %
%%%%%%%%%%%%%%%%%%%%%%%%%%%%%%%%%%%%%%%%%%%%%%%%%%%%%%%%%%%%%%%%%%%%%%%%%%%%%%%%
\section{Traditional communication mechanisms in cluster systems}
\label{sec:communication}
\todo{Examples for communication intensive simulations}
Communication plays a central role in complex simulation
applications. The calculation of large datasets in a
acceptable amount of time is not possible on a single compute
device anymore.

 Thus the simulation need to be decomposed in
subdomains and distributed to more compute devices.  The Subdomains
are usually not independent from each other, they need to exchange
data. There are several methods to exchange data between the nodes.

The most simple way is to let the nodes communicate directly with each
other in a peer to peer manner.  This can be done by low level protocols like TCP, UDP, IP
using Berkeley sockets (\ref{sec:tcp_udp_ip}), which is communication
in the UNIX way; everything is a file or file descriptor. But Sockets
where not designed for the usage in a high performance computing
environment in mind.

\todo{boost::asio, ZMQ}

Active messages \cite{ref:am} is a slight departure from the classical
send-receive model. It is based on one-sided communication operations,
which transfer messages regardless of the current activity of the
receiver process. It was designed to reduce the impact of
communication overhead on application performance.

A more modern and abstract model is the Message Passing Interface,
what is described in section \ref{sec:mpi}.

An other way is to let the compute cluster look like a single large
parallel computer. The operating system MOSIX \cite{ref:mosix} is a
linux based multi-computer OS (MOS), that provides a single-systems
image\cite{ref:single_system_image} as if using one computer with
multiple CPUs. The Parallel Virtual Machine\cite{ref:pvm} is a
software package that permits a heterogeneous collection of Unix
and/or Windows computers hooked together by a network to be used as a
single computer.

These distributed shared architectures make it possible to use
standard and simple parallization schemas like Pthreads or OpenMP.
Communication between the nodes does not have to be modeld explicitly,
thus it is easy to use for non computer scientiest. But with this kind
of higher abstraction allways comes a loose of control and therefore
also efficiency. Also the underlying compute cluster has to provide
such an architecture, which limits the application to this
kind of clusters.

\subsection{Message passing intferface (MPI)}
\label{sec:mpi}
The Message Passing Interface (MPI) is a standardized and portable
message-passing specification, specified by the Message Passing
Interface Forum \cite{ref:mpi_specification}. It has become the de
facto standard for communication in cluster systems.

The specification is language independent and its implementations are
available on virtually every parallel computer system.  Popular
implementations are MPICH and OpenMPI. MPICH2 implements the MPI-2.1
specification. OpenMPI has the goal to be full MPI-3 standard conform.

The MPI implementations are implemented on top of existing communication
abstractraction provided by the underlying cluster system (sockets,
AM, etc.)

The first version of the specification, known as MPI-1, was completed
in 1994 making use of the most attractive features of allready
existing message passing systems.  Further improvements followed 1997
by version MPI-2 and 2008 by version MPI-3. There might be some
confusion about the MPI specification versions. Version numbers follow
the policy that major versions like MPI-2 or MPI-3 come with new
features to the MPI standard and minor versions like MPI-1.1 or
MPI-1.2 come with clarifiacations of allready existing features. In
the following the key features of the varying MPI versions are listed:

\begin{itemize}
  \item MPI-1 (1994)
    \begin{itemize}
      \item Basic point-to-point communication
      \item Collectives
      \item Derived datatypes
    \end{itemize}
  \item MPI-2 (1997)
    \begin{itemize}
      \item Parallel I/O
      \item Remote memory operations (one-sided)
      \item Dynamic process management
    \end{itemize}
  \item MPI-3 (1998)
    \begin{itemize}
      \item Nonblocking collectives
      \item Improved one-sided communication interface
    \end{itemize}
\end{itemize}

An MPI programm is a set of autonomous processes, executing their own
code, in an MIMD style \cite{Flynn:1972:COE:1952456.1952459}. Each
process rests in his own adress space and communicates via MPI
communication primitives with other processes in the same MPI
programm.

Processes are identified according to their relative rank in a group
by consecutive integers from 0 to groupsize minus 1. Which the initial
group compromises the full set of processes with ranks in the range 0
to number of processes minus one. A typical MPI program run is composed from
the following steps:

\begin{enumerate}
\item Start MPI programm
\item Init MPI environment
\item Ask for own rank
\item Ask for number of ranks overall
\item Depend on own rank and number of overall processes do some task
  (varying flow of control between ranks)
\item Communicate with other processes
\item Finalize MPI environment
\item Exit MPI programm
\end{enumerate}

MPI provides a large and versatile set of routines, that fit the needs
of the parallel programming community. At its most basic, MPI has
functions for sending and receiving a message.  These functions are
available both blocking and nonblocking.

\todo{Describe collective in own section!}
Further a variaty of collective operations that involve a group of
processes are defined. All members of the group have to paticipate to
sucessfully execute the collective operation.  The result of the
operation is either received by one process, the so called root, or by
all processes of a group.

A collective operation can only be successful, when all peers of a
context are participating. If only one peer does not execute the
operation then the application is stuck. It will only continue when
this peer also executes this collective operation.

\begin{description}
\item[Root] \hfill \\
  The root peer receives the result of a collective operation. For some
  collective operations (e.g allGather, allScatter, allReduce) all peers
  of a context receive the result of the operation.
\end{description}


%%%%%%%%%%%%%%%%%%%%%%%%%%%%%%%%%%%%%%%%%%%%%%%%%%%%%%%%%%%%%%%%%%%%%%%%%%%%%%%%
%                                                                              %
% JUNGLE COMPUTING                                                             %
%                                                                              %
%%%%%%%%%%%%%%%%%%%%%%%%%%%%%%%%%%%%%%%%%%%%%%%%%%%%%%%%%%%%%%%%%%%%%%%%%%%%%%%%
\section{Jungle computing}

Jungle computing can be seen as a further evolution step after
grid computing. The initial intention of grid computing was
the request to bundle the tremendous compute power of hpc
systems over the internet\cite{ref:grid}. It was meant to an
easy to use system over a set of distributed ressources, which
delivers potentially efficient computing, even at a worldwide scale.
The hook with grid computing is, that every compute systems has its
own administration environment, submit system, communication middleware,
and programming interface. 

Even though every system viewed in isolation is programmable by using
platform specific code, write applications that interconects several
compute systems can be a hard challange. And with the increasing
desire for speed, sclability and flexibility in many scientific research
domains, there is demand for easy end-user access to multiple and potentially
very diverse platforms.

``A true computing jungle'' where the words by which F. J. Seinastra
introduced such hetergeneous systems as jungle computing
systems\cite{ref:jungle}. It is defined as simultaneously using a
combination of heterogeneous, hierarchical and distributed computing
ressources.

This definition describes the undergoing revolutionary change in 
hpc area very well. On one hand, computations will be distributed
to several compute clusters and on the other hand the underlying
hardware of compute nodes will change to heterogeneous hardware in
the years to come. The efficient programming by scientist of such systems, 
und thereby the efficient mapping of computational problems onto
the jungle has become enormously difficult to maintain and overview.


\begin{itemize}
  \item Jungle computing requirements
\end{itemize}


%%%%%%%%%%%%%%%%%%%%%%%%%%%%%%%%%%%%%%%%%%%%%%%%%%%%%%%%%%%%%%%%%%%%%%%%%%%%%%%%
%                                                                              %
% PIConGPU - THE APPLICATION                                                   %
%                                                                              %
%%%%%%%%%%%%%%%%%%%%%%%%%%%%%%%%%%%%%%%%%%%%%%%%%%%%%%%%%%%%%%%%%%%%%%%%%%%%%%%%
\section{PIConGPU - the application}
\label{sec:picongpu}
PIConGPU is a relativistic particle-in-cell (PIC) code designed for
simulating laser-plasma interactions \cite{ref:picongpu}. The
application is optimized for the execution on cluster systems with
NVIDIA GPU-accelerated\ref{sec:accel} nodes, whereby the host CPUs are
only utilized for memory transfers between CPU and GPU and for
internode communication via MPI (Section \ref{sec:mpi}).

The Computational power of a single GPU is limited in regard of clock
frequency and memory size by physical laws.

Therefore, The simulation volume need to be divided into
subdomains\ref{sec:domain_decomposition}. Each subdomain is mapped to
its own GPU. One can imagine that the simulation volume is a cuboid
and the subdomains are filling up that cube with smaller cuboids,
also called cells.

Particles, fields etc. can cross the border of the
cells and therfore data between cells has to be
exchanged. The data still is not transfered directly between 
the GPUs, thus it has to take the detour via host and network, instead
of just the network.

Individual cells are communicating mainly with the next
neighboring cells. To decouple communication and computation,
nonblocking variant of this basic operations are used.

\todo{find out why collective operations are used} There are also some
collective operations like allgather, gather gatherv, allreduce and
reduce used.


%%%%%%%%%%%%%%%%%%%%%%%%%%%%%%%%%%%%%%%%%%%%%%%%%%%%%%%%%%%%%%%%%%%%%%%%%%%%%%%%
%                                                                              %
% RELATED WORK                                                                 %
%                                                                              %
%%%%%%%%%%%%%%%%%%%%%%%%%%%%%%%%%%%%%%%%%%%%%%%%%%%%%%%%%%%%%%%%%%%%%%%%%%%%%%%%
\section{Related work}

\begin{itemize}
  \item ibis
  \item ProActive
  \item Constellation
  \item European grid initiative (EGI)
  \item ComplexHPC: Open European Network for High-Performance Computing in Complex Environments (COST)
\end{itemize}


\cleardoublepage

%%% Local Variables:
%%% TeX-master: "diplom"
%%% End:
