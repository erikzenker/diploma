\chapter{Conclusion And Outlook}
\label{sec:conclusion}

%  Schlußfolgerungen, Fragen, Ausblicke

% Dieses Kapitel ist sicherlich das am Schwierigsten zu schreibende. Es
% dient einer gerafften Zusammenfassung dessen, was man gelernt hat. Es
% ist möglicherweise gespickt von Rückwärtsverweisen in den Text, um dem
% faulen aber interessierten Leser (der Regelfall) doch noch einmal die
% Chance zu geben, sich etwas fundierter weiterzubilden. Manche guten
% Arbeiten werfen mehr Probleme auf als sie lösen. Dies darf man ruhig
% zugeben und diskutieren. Man kann gegebenenfalls auch schreiben, was
% man in dieser Sache noch zu tun gedenkt oder den Nachfolgern ein paar
% Tips geben. Aber man sollte nicht um jeden Preis Fragen, die gar nicht
% da sind, mit Gewalt aufbringen und dem Leser suggerieren, wie
% weitsichtig man doch ist. Dieses Kapitel muß kurz sein, damit es
% gelesen wird.


\begin{itemize}

  % What to do with the developed system
  
  % What was developed - General Design
\item A system was developed, which provides an intermediate layer in
  between an application and existing communication libraries. This
  intermediate layer maps common communication processes onto this
  existing communication library and is build from three components:
  First, the communication abstraction layer (CAL) that provides an
  general interface for already existing communication libraries by
  adapters. Second, a graph models the communication topology of an
  application. This graphs can be generated from pre-defined graph
  generation functions and be reused again. Third, the mapping of the
  modeled graph onto the abstract communication layer constructs a
  graph-based virtual overlay network (GVON), which allows to
  communicate on graph-basis. This mapping is constructed at run-time
  and can be modified if it is required by the applications algorithm.

  % What was developed - Implementation
\item The CAL is implemented through policy based design, which
  provides the possibility to exchange the underlying communication
  library by adapters. These adapters are configurable at compile-time
  which minimizes run-time overhead with respect to the utilized
  communication library.  The Boost Graph Library was utilized as
  back-end to implement the graph. This library provides a rich set of
  graph algorithms and can be deployed comfortable into C++ projects.
  The combination of CAL and graph enables the GVON to provide
  communication mechanism on basis of the modeled graph.
  
  % What was learned from evaluation
\item An adapter based on MPI as underlying communication library was
  implemented and tested in synthetic, real world and flexibility
  benchmarks. The evaluation of these benchmarks turned out that an
  implementation based on the system provides in the most cases
  negligible overhead with respect to an MPI
  implementation. Furthermore has the system demonstrated the
  execution of a Game of Life simulation with varying number of peers
  at run-time.  Only in some cases is the look up of graph information
  a source of overhead, which should be addresses in future
  optimizations.
  
  % What are the advantages of the system
\item  The system offers the advantages that the CAL can be adapted to the utilized
  computing system by exchanging the underlying adapter. By providing
  an explicit mapping from vertices of the graph to peers of the CAL
  are application based on the system very flexible because a
  modification of this mapping possible at run-time.

  % Outlook  
\item The system shows acceptable performance for deployment in
  simulation applications such as PIConGPU. The GVON could completely
  replace the existing communication code of this simulation and
  furthermore provide the ability to comfortable exchange
  communication library and topology.

\item The system can be enhanced by load balancing and fault
  tolerance techniques required by the upcoming exascale systems.

\item Future work should focus on the implementation of further
  adapter. These adapters would expand the field of application
  of the developed system. 
  
\end{itemize}

\cleardoublepage

%%% Local Variables:
%%% TeX-master: "diplom"
%%% End:
