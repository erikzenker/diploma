\chapter{Conclusion And Outlook}
\label{sec:conclusion}

%  Schlußfolgerungen, Fragen, Ausblicke

% Dieses Kapitel ist sicherlich das am Schwierigsten zu schreibende. Es
% dient einer gerafften Zusammenfassung dessen, was man gelernt hat. Es
% ist möglicherweise gespickt von Rückwärtsverweisen in den Text, um dem
% faulen aber interessierten Leser (der Regelfall) doch noch einmal die
% Chance zu geben, sich etwas fundierter weiterzubilden. Manche guten
% Arbeiten werfen mehr Probleme auf als sie lösen. Dies darf man ruhig
% zugeben und diskutieren. Man kann gegebenenfalls auch schreiben, was
% man in dieser Sache noch zu tun gedenkt oder den Nachfolgern ein paar
% Tips geben. Aber man sollte nicht um jeden Preis Fragen, die gar nicht
% da sind, mit Gewalt aufbringen und dem Leser suggerieren, wie
% weitsichtig man doch ist. Dieses Kapitel muß kurz sein, damit es
% gelesen wird.


  % What to do with the developed system
  
  % What was developed - General Design
A system was developed, which provides an intermediate layer in
between an application and an existing communication library. This
intermediate layer maps common communication processes onto this
existing communication library and is build from three components:
First, the communication abstraction layer (CAL) that provides an
general communication interface for existing communication libraries
which are exchangeable through adapters.  Second, a graph that models the
communication topology of an application.  Third, the graph-based
virtual overlay network (GVON) that provides a mapping from the
modeled graph onto the CAL and therefore allows communication on
basis of the graph.

% What was developed - Implementation

Policy based design allows the CAL to exchange the utilized
communication library by template programming, which offers
flexibility and minimizes the run-time overhead by compile-time
optimizations.  The Boost Graph Library was utilized as back-end to
implement the graph class. This library provides a rich set of graph
algorithms and can be deployed into C++ projects comfortably.

%The combination of CAL and graph enables the GVON to provide
%communication mechanism on basis of the modeled graph.

% What are the advantages of the system

The system offers the advantage that the CAL can be adapted to the
utilized computing system by exchanging the underlying adapter.  The
graphs can be generated from predefined graph generation functions
which offer the possibility to reuse these graphs. Furthermore, are
vertices of these graphs explicitly mapped onto peers of the CAL
within the GVON at run-time. This mapping can be optimized to
exploit maximum performance and be modified if it is required by the
applications algorithm.

% What was learned from evaluation

An adapter based on the Message Passing Interface (MPI) as
underlying communication library was implemented and tested in
synthetic, real world and flexibility benchmarks. The evaluation of
these benchmarks turned out that an implementation based on the
system provides in the most cases negligible overhead with respect
to an MPI implementation.  The system has demonstrated the variation
of the number of peers during run-time for the execution of a Game
of Life simulation. Which shows that the developed system can be the
foundation to implement load balancing and fault tolerance mechanism
on top.  Only in some cases is the look up of graph information a
source of overhead, which should be addressed in future
optimizations.

% Outlook  

The system shows acceptable performance for deployment in
simulation applications such as PIConGPU. The GVON could completely
replace the existing communication code of this simulation and
furthermore provide the ability to comfortable exchange
communication library and communication topology.

Future work should focus on the implementation of further adapters. So
could an accelerator aware adapter allow direct data exchange between
accelerator devices without the detour of the CPU. Connecting peers
over an internet socket based adapter would create a distributed
system by keeping the same interface.  These adapters would expand the
field of application of the developed system and the applications
based on the system.  Furthermore, should the system be enhanced by
load balancing and fault tolerance techniques required by the upcoming
exascale systems.
  

\cleardoublepage

%%% Local Variables:
%%% TeX-master: "diplom"
%%% End:
