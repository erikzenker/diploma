\chapter{Conclusion}
\label{sec:conclusion}

%  Schlußfolgerungen, Fragen, Ausblicke

% Dieses Kapitel ist sicherlich das am Schwierigsten zu schreibende. Es
% dient einer gerafften Zusammenfassung dessen, was man gelernt hat. Es
% ist möglicherweise gespickt von Rückwärtsverweisen in den Text, um dem
% faulen aber interessierten Leser (der Regelfall) doch noch einmal die
% Chance zu geben, sich etwas fundierter weiterzubilden. Manche guten
% Arbeiten werfen mehr Probleme auf als sie lösen. Dies darf man ruhig
% zugeben und diskutieren. Man kann gegebenenfalls auch schreiben, was
% man in dieser Sache noch zu tun gedenkt oder den Nachfolgern ein paar
% Tips geben. Aber man sollte nicht um jeden Preis Fragen, die gar nicht
% da sind, mit Gewalt aufbringen und dem Leser suggerieren, wie
% weitsichtig man doch ist. Dieses Kapitel muß kurz sein, damit es
% gelesen wird.


% Motivation
A system was developed with the requirements for an explicit modeling
of the communication topologies of an application and an explicit
mapping of this topologies to the hardware topolgy at run-time. The
modeling and mapping is necessary because modern simulations show
heterogeneous behavior both in horizontal and vertical direction.

To address these heterogeneities provides the developed system an
intermediate layer in between an application and an existing
communication library.  This layer maps common communication processes to a
existing communication library and is build from three components:
First, the communication abstraction layer (CAL) that provides an
general communication interface for existing communication libraries
which are exchangeable through adapters.  Second, a graph that models
the communication topology of an application.  Third, the graph-based
virtual overlay network (GVON) that provides a mapping from the
modeled graph to the CAL at run-time and provides the possibilty to
change that mapping during the simulation execution.  The
communication topology modeled by the graph is used to provide
communication operations on basis of the graph.

% What was developed - Implementation
Policy based design allows the CAL to exchange the utilized
communication library by template programming, which offers
flexibility and minimizes the run-time overhead by compile-time
optimizations.  The Boost Graph Library was utilized as back-end to
implement the graph class. This library provides a rich set of graph
algorithms and can be deployed into C++ projects comfortably.

% What are the advantages of the system
The system offers the advantage that the CAL can be adapted to the
utilized computing system by exchanging the underlying adapter.  The
graphs can be generated from predefined graph generation functions
which offer the possibility to reuse these graphs. Furthermore, are
vertices of these graphs explicitly mapped onto peers of the CAL
within the GVON at run-time. This mapping can be optimized to exploit
maximum performance and be modified at run-time if it is required by
the applications algorithm.

% What was learned from evaluation
An adapter based on the Message Passing Interface (MPI) as underlying
communication library was implemented and tested in synthetic, real
world and flexibility benchmarks. The evaluation of these benchmarks
turned out that the overhead with respect to an MPI implementation is
negligible.  The system has demonstrated the variation of the number
of peers during run-time for the execution of a Game of Life
simulation also with negligible overhead. Only in some extremely unrealistic
cases is the look-up of graph information a source of overhead, which
should be addressed in future optimizations.

% Finally
The evaluation has shown that the developed system can be used to
implement heterogeneous applications on heterogeneous compute systems,
load balancing and fault tolerance mechanisms. This was achieved
through a consistent concept which allows to model an application by
an abstract description. This description is provided by a very
lightweight system, yet powerful.


\chapter{Outlook}
\label{sec:outlook}

% Outlook  

The system shows acceptable performance. Thus, the next step is the
deployment into the simulation PIConGPU. The GVON could completely
replace the existing communication code of this simulation and
furthermore provide the ability to comfortable exchange communication
library and communication topology.

Future work should focus on the implementation of further adapters. So
could an accelerator-aware adapter allow direct data exchange between
accelerator devices without the detour of the CPU. Connecting peers
over an internet socket based adapter would create a distributed
system by keeping the same interface.  These adapters would expand the
field of application of the developed system and the applications
based on the system.  

Load balancing and fault tolerance are important topics
for the upcoming exascale system. The system could be utilized in
these areas especially the remapping at run-time with respect to
load and energy consumption information.

All the developed techniques allow a unified description of
heterogeneous computations, load balancing and fault tolerance
and consequently clarify their similarities. Furthermore, reduce
these techniques the complexity of these problem significantly.


  

\cleardoublepage

%%% Local Variables:
%%% TeX-master: "diplom"
%%% End:
