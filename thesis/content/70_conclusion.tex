\chapter{Conclusion}
\label{sec:conclusion}

%  Schlußfolgerungen, Fragen, Ausblicke

% Dieses Kapitel ist sicherlich das am Schwierigsten zu schreibende. Es
% dient einer gerafften Zusammenfassung dessen, was man gelernt hat. Es
% ist möglicherweise gespickt von Rückwärtsverweisen in den Text, um dem
% faulen aber interessierten Leser (der Regelfall) doch noch einmal die
% Chance zu geben, sich etwas fundierter weiterzubilden. Manche guten
% Arbeiten werfen mehr Probleme auf als sie lösen. Dies darf man ruhig
% zugeben und diskutieren. Man kann gegebenenfalls auch schreiben, was
% man in dieser Sache noch zu tun gedenkt oder den Nachfolgern ein paar
% Tips geben. Aber man sollte nicht um jeden Preis Fragen, die gar nicht
% da sind, mit Gewalt aufbringen und dem Leser suggerieren, wie
% weitsichtig man doch ist. Dieses Kapitel muß kurz sein, damit es
% gelesen wird.


% Motivation
A system was developed with the requirements for an explicit modeling
of the communication topologies of a simulation and an explicit
mapping of these topologies to the hardware topologies of the
computing system at run-time. This is necessary since modern
simulations show heterogeneous behavior both in horizontal and
vertical direction.

To address these heterogeneities, the developed system provides
modeling and mapping through an intermediate layer in between a
simulation application and an existing communication library.  This
layer maps common communication processes to an existing communication
library and is build from three components: First, the communication
abstraction layer that provides an general communication interface for
existing communication libraries which are exchangeable through
adapters.  Second, a graph that models the communication topology of a
simulation.  Third, the graph-based virtual overlay network that
provides a mapping from the modeled graph to the CAL at run-time.
Furthermore, the GVON enables to utilize the communication topology
modeled by the graph to provide communication operations on basis of
this graph.

% What was developed - Implementation
Policy based design allows the CAL to exchange the utilized
communication library by template programming, which offers
flexibility and minimizes the run-time overhead by compile-time
optimizations.  The Boost Graph Library was utilized as back-end to
implement the graph class. This library provides a rich set of graph
algorithms and comfortable deployment into C++ projects.

% What are the advantages of the system
The system offers the advantage that the CAL can be adapted to the
utilized computing system by exchanging the underlying adapter.  The
graphs can be generated from predefined graph generation functions,
which provides the possibility to reuse these graphs. Furthermore, 
vertices of these graphs are explicitly mapped to peers of the CAL
within the GVON at run-time. This mapping can be optimized to exploit
maximum performance and be modified at run-time if this is required
during execution.

% What was learned from evaluation
An adapter based on MPI was implemented and tested in synthetic,
real world, and flexibility benchmarks. The evaluation of these
benchmarks turned out that the overhead with respect to an MPI
implementation is negligible.  The system has demonstrated the
variation of the number of peers during run-time for the execution of
a Game of Life simulation also with negligible overhead. Only the
look-up of graph information is in some extreme corner cases a
source of overhead.

% Finally
The evaluation has shown that the developed system can be used to
implement heterogeneous applications, load balancing and fault
tolerance mechanisms. This was achieved through a consistent concept
which allows to model an application by an abstract description. This
description is provided by a very lightweight, yet powerful system.


\chapter{Outlook}
\label{sec:outlook}

% Outlook  

The developed abstract interface shows negligible run-time overhead
with respect to the reference communication library MPI. The
deployment into the previously mentioned simulation PIConGPU is the
next step.  The GVON interface would replace the existing
communication code completely and furthermore provide the ability to
exchange the underlying communication library and the utilized
communication topology.

The implementation of further adapters would expand the field of
application. So would an accelerator-aware adapter allow for direct data
exchange between accelerator devices without the detour of the CPU.
An adapter based on internet sockets would provide a distributed
computing environment similar to grids.

Load balancing and fault tolerance are important topics for the
upcoming exascale systems.  Especially the remapping feature at
run-time with respect to workload and energy consumption would be
reasons for a deployment which allows for a scaleable utilization of
the hardware ressources. All these problems are solved through a
unified description which clarifies their similarities and therefore
reduces the complexity of such problems significantly.

\cleardoublepage

%%% Local Variables:
%%% TeX-master: "diplom"
%%% End:
